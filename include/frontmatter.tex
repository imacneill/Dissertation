% No symbols, formulas, superscripts, or Greek letters are allowed
% in your title.
\title{Measurement of top quark-antiquark pair production in association with a Z boson with a trilepton final state in pp collisions at 8 TeV center of mass energy}

\author{Ian Christopher MacNeill}
\degreeyear{2015}

% Master's Degree theses will NOT be formatted properly with this file.
\degreetitle{Doctor of Philosophy} 

\field{Physics}
\chair{Professor Avraham Yagil}

%  The rest of the committee members  must be alphabetized by last name.
\othermembers{
Professor Claudio Campagnari\\ 
Professor Aneesh Manohar\\
Professor George Tynan\\
Professor Frank W\"urthwein\\
}
\numberofmembers{5} % |chair| + |cochair| + |othermembers|


\begin{frontmatter}
\makefrontmatter                                                               

%% ----------------------------------------------------------------------- %%
%% DEDICATION
%% ----------------------------------------------------------------------- %%

\begin{dedication}                                                             
dedication
\end{dedication}                                                               
\clearpage 

%% ----------------------------------------------------------------------- %%
%% EPIGRAPH
%% ----------------------------------------------------------------------- %%

%  The same choices that applied to the dedication apply here.
% \begin{epigraph} % The style file will position the text for you.              
%   \it{Mon seul d\'esir est de m'enrichir de nouvelles pens\'ees exaltantes.} \\
%   ---Ren\'e Magritte
% \end{epigraph}                                                                 
\begin{myepigraph} % You position the text yourself.                           
  \vfil                                                                        
  \vfil 
  \hfill {\it Numquam aliud natura, aliud sapientia dicit.} \\
  \vfil 
  \noindent {\it Never does nature say one thing and wisdom say another.} \hfill \\
  \vfil 
  \hfill ---Juvenal
  \vfil 
\end{myepigraph}                                                               

\tableofcontents
\listoffigures  % Uncomment if you have any Figures                            
\listoftables   % Uncomment if you have any Tables                             

\begin{acknowledgements}                                                       
Well\ldots it's finally here. This work is finished, simultaneously seeming to pass slowly and at the same time taking me entirely by surprise that it's done. I have to give much of the credit for the completion of this work to my advisor, Avi Yagil, and post doc task master, Slava Krutelyov. They helped immensely to raise the quality of this work and the quality of this Ph.D. student. Other collaborators deserve to be thanked. Each provided their own lessons, outlooks, and way of doing things. If I did not learn from them, it is certainly not their fault. Frank W\"urthwein, Frank Golf, Giuseppe Cerati, Dave Evans, Ben Hooberman, Verena Martinez, Ryan Kelley, Vince Welke, and Warren Andrews.  I'd like to point out a special thanks to Frank and Warren for getting me started, and Ryan for being a sounding board and technical expert near the end of the work. Finally, Boris Mangano, Amanda Deisher, Lukas B\"ani, and Matthew Walker were excellent collaborators on the combined cross section and extremely patient dealing with someone not working at CERN. \\

I suppose I should blame my parents (Fletcher and Martha), or at least thank them, for starting me on this road. They both encouraged me to ask questions and instilled a sense of wonder and exploration in me directed towards the world we live in. I learned to love reading, sports, building things with my hands, and figuring out how things work because of their influence. To be honest, there's plenty of blame to go around here. My grandparents, none of whom lived to see me finish this work, started early with teaching me how to make plant clippings, play scrabble and chess, and to love cooking. Yes, many of the things listed don't seem related, but being a physicist is about more than just knowing math and sitting in a windowless office. It comes down to wanting to understand how something works and continuing to ask questions until you do. All of them deserve a great deal of thanks for teaching me the rewards of striving to understand.\\

For the past few years, I have had some wonderful companions who have figuratively and, unfortunately, literally propped me up. My partner, Wyn, deserves a lot of credit for loading all of our possessions into a truck and blindly driving across the country to a strange new land, California, with me. We've both sacrificed out here by being nearly 3000 miles away from our families, but have also made a very rewarding life for ourselves. Wyn has been unflaggingly encouraging and supportive. Wyn also helped to usher in two extra companions, Freddie and Fiona. As I write, Freddie is glued to my side. He's either encouraging me to write, or making sure I don't forget to feed him. It's hard to say which. Fiona was a cranky bitch (literally), but she also knew just the right way to show love and attention before she passed away. Until that point, she helped keep me company while I worked late and helped to relieve stress and provide an outlet for my energy other than physics. Despite her insistence to the contrary, she was a terrible coder. Freddie still endeavors to fill this role.\\

There are many other people who have helped me while I've lived out here and worked on this Ph.D. They include my aunt, Susan, who took us in the first night we arrived and has been a helpful and a familiar face ever since then. Credit goes to all of my friends on the cycling team, who have given me something to do other than sit in a windowless office. The beer we've had together and the bikes we've ridden (\ldots and crashed) together have kept me sane and balanced these past few years. Finally, I'd like to thank the rest of the graduate students at UCSD. You have been a great group of people to learn with.\\

\end{acknowledgements}                                                         

\begin{vitapage}                                                               
\begin{vita}                                                                   
  \item[2009] B.~A. in Physics, University of Pennsylvania
  \item[2011] M.~S. in Physics, University of California, San Diego
  \item[2015] Ph.~D. in Physics, University of California, San Diego       
\end{vita}                                                                     
\begin{publications}                                                           
 \item Measurement of top quark-antiquark pair production in association with a W or Z boson in pp collisions at $\sqrt{s}$ = 8 TeV, {\it CMS Collaboration}, The European Physical Journal C 74 (2014) No 9, doi:10.1140/epjc/s10052-014-3060-7, [arXiv:1406.7830 [hep-ex]]


 
 %\item Search for new physics in events with same-sign dileptons and b jets in pp collisions at $\sqrt{s} = 8$ TeV, {\it CMS Collaboration}, JHEP 1303 (2013) 037 [arXiv:1212.6194 [hep-ex]] % HCP
  %\item 2013!!!Search for new physics in events with same-sign dileptons and b jets in pp collisions at $\sqrt{s} = 8$ TeV, {\it CMS Collaboration}, JHEP 1303 (2013) 037 [arXiv:1212.6194 [hep-ex]] % 2013
\end{publications}                                                             
\end{vitapage}                                                                 
                                                                               

%% ABSTRACT
%  Doctoral dissertation abstracts should not exceed 350 words. 
%   The abstract may continue to a second page if necessary.
\begin{abstract}
A search for Z boson production with associated top-antitop pairs is performed with a final state which contains trileptons and b-tagged jets. The measurement is performed on the complete dataset of pp collisions at a center-of-mass energy of 8 TeV collected at the CMS detector, for a total of \lumi = 19.5~\fbinv. A cross section is calculated for this final state and compared to the standard model prediction. The cross section for \ttZ production is measured to be $\sigma=194 _{-89} ^{+105}$ \ fb. The measured cross section to theoretical cross section ratio is $0.94 _{-0.43} ^{+0.51}$. The process is measured with a significance of 2.33. This measurement is then combined with an outside 4 lepton channel measurement for an equivalent cross section measurement of $\sigma=200 _{-76} ^{+90} $ \ fb and a significance of 3.1.
 

\end{abstract}


\end{frontmatter}
