% --------------------------------------------------------------------------- %
% --------------------------------------------------------------------------- %
\chapter{Summary and Conclusions}
\label {ch:conclusion}
% --------------------------------------------------------------------------- %
% --------------------------------------------------------------------------- %

This dissertation presents the measurement of the cross section of associated production of top anti-top pairs with Z bosons at $\sqrt{s} = 8 \TeV$. The measurement is an inclusive search performed in a tri-lepton final state. In \intLumi of data, the cross section for \ttZ \ production is measured as $\sigma=194 _{-89} ^{+105}$ \ fb with a significance of 2.33. Finally, the ratio of measured to theoretical cross section is $0.94_{-0.43} ^{+0.51}$.\\

Additionally the results of a combination of this work with the work of other groups has been shown and reports a \ttZ cross section of 
$200 ^{+80}_{-70} \textrm{(stat.)} ^{+40}_{-30} \textrm{(syst.)}$~fb, with a significance of $3.1$ standard deviations. The combined cross section of \ttZ and \ttW (\ttV) is reported as $\sigma_{\ttV} = 380 ^{+100}_{-90} \textrm{(stat.)} ^{+80}_{-70} \textrm{(syst.)}$~fb.\\

\ttZ production will continue to be an important topic as the CMS Experiment looks towards future runs of the LHC at 13 and 14 \TeV center-of-mass energy collisions. The work done here helps to experimentally constrain and measure the \ttZ production cross section which, aside from the indepentent achievement of measuring a SM phenomenon, will help to guide MC simulations used for estimating backgrounds and desiging background subtraction techniques for New Physics searches. \ttZ searches at higher energy and with more integrated luminosity measured in the future, will prove invaluable to help measure the weak coupling interactions of the top quark as one of the main diagrams contributing to \ttZ production involves direct coupling of the Z boson to the top quark.\\ 
