\chapter{Real and Simulated Collision Sources}
\label{ch:samples}
This document presents results obtained from data of proton-proton collisions at 8 TeV center of mass energy. A total luminosity of \intLumi \ is used, which corresponds to the totality of data to be measured at this center of mass energy at the LHC. The data was collected by the CMS detector through the end of 2012. This analysis relies on a mixture of data driven methods and MC simulation to estimate backgrounds. Additionally, signal events are generated for use in this analysis. The following section discusses the sources and luminosity of the various data sets from various data taking sessions as well as the many Monte Carlo simulation samples used to help investigate the relevant processes and estimate backgrounds.\\


\section{Collision Data Samples}
		
	
	Data used is a combination of prompt (Run2012C v2 and D) and re-reco (Run2012A, B, part of C, and part of D) data.  
Run2012A comes from both the 13July re-reco and 06August recovery campaigns while Run2012B is only from the former.  
Data from Run2012C v1, corresponding to ~0.5 \fbinv, comes from the 24Aug re-reco while v2 is from prompt. Rounding out Run2012C is an addition Dec11 Ecal Recovery rereco for 0.133 \fbinv. Additionally, data for Run2012D comes from prompt.
    
Only events from certified data-taking periods are used.  
The selection of good run and luminosity sections comes from a combination of prompt and re-reco certification.  
The certified dataset considered in this document covers runs up to 208686 inclusive and corresponds to \intLumi \ and an associated error of 2.6\%~\cite{lumi12up}. 


	
	
	
Signal events are selected from the datasets and luminosity totals listed in Table~\ref{tab:DilDsets_lumi} and run ranges are listed in Appendix~\ref{sec:data_details:datasets}.  The fake rate measurement is performed using those listed in Table~\ref{tab:FRDsets}.  Runs found in the Run2012A 06Aug2012 recover are excluded from the Run2012A 13Jul2012 re-reco dataset. 

%\begin{table}[hbt]
%\begin{center}
%\begin{tabular}{lcc}\hline\hline
%Name		& Run Range & Luminosity ($fb^{-1}$) \\ \hline
%\verb=/DoubleMu/Run2012A-recover-06Aug2012-v1/AOD=                 & 190782 - 190949 &  0.081 \\ 
%\verb=/DoubleMu/Run2012A-13Jul2012-v1/AOD=                                  &  190456 - 193621       & 0.796               \\ 
%\verb=/DoubleMu/Run2012B-13Jul2012-v4/AOD=                                  &  193834 - 196531        & 4.412             \\ 
%\verb=/DoubleMu/Run2012C-24Aug2012-v1/AOD=                                &  197770 - 198913  & 0.473\\  
%\verb=/DoubleMu/Run2012C-PromptReco-v2/AOD=                               &  198934 - 203755     & 6.330                \\ 
%\verb=/SingleMu/Run2012C-EcalRecover_11Dec2012-v1/AOD=          & 201 191 - 201 191 & 0.133\\
%\verb=/DoubleMu/Run2012D-PromptReco-v1/AOD=                               &  203768 - 208913  &  7.295 \\
%%\verb=/DoubleMu/Run2012D-16Jan2013-v1/AOD=                                 &  207883 - 208307  \\
%
%\verb=/DoubleElectron/Run2012A-recover-06Aug2012-v1/AOD=         &    190782 - 190949     & 0.081              \\ 
%\verb=/DoubleElectron/Run2012A-13Jul2012-v1/AOD=                         & 190456 - 193621   & 0.796                    \\ 
%\verb=/DoubleElectron/Run2012B-13Jul2012-v1/AOD=                         &  193834 - 196531  & 4.412\\ 
%\verb=/DoubleElectron/Run2012C-24Aug2012-v1/AOD=                       &  197770 - 198913    & 0.473                 \\ 
%\verb=/DoubleElectron/Run2012C-PromptReco-v2/AOD=                     &   198934 - 203755     & 6.330             \\ 
%\verb=/SingleMu/Run2012C-EcalRecover_11Dec2012-v1/AOD=          & 201 191 - 201 191 & 0.133\\
%\verb=/DoubleElectron/Run2012D-PromptReco-v1/AOD=                      &  203768 - 208913  &  7.295 \\
%%\verb=/DoubleElectron/Run2012D-16Jan2013-v2/AOD=                        &   207883 - 208307 \\
%
%\verb=/MuEG/Run2012A-recover-06Aug2012-v1/AOD=                          &      190782 - 190949     & 0.081            \\ 
%\verb=/MuEG/Run2012A-13Jul2012-v1/AOD=                                          &  190456 -193621         & 0.796             \\ 
%\verb=/MuEG/Run2012B-13Jul2012-v1/AOD=                                         &  193834 -196531      & 4.412 \\ 
%\verb=/MuEG/Run2012C-24Aug2012-v1/AOD=                                      &   197770 - 198913     & 0.473               \\ 
%\verb=/MuEG/Run2012C-PromptReco-v2/AOD=                                     &   198934 - 203755      & 6.330              \\ 
%\verb=/SingleMu/Run2012C-EcalRecover_11Dec2012-v1/AOD=          & 201 191 - 201 191 & 0.133\\
%\verb=/MuEG/Run2012D-PromptReco-v1/AOD=                                     &  203768 - 208913  &  7.295 \\
%%\verb=/MuEG/Run2012D-16Jan2013-v2/AOD=                                       &  203768 - 208307 \\
%
%\verb=/SingleMu/Run2012A-recover-06Aug2012-v1/AOD=                    &   190782 - 190949          & 0.081          \\ 
%\verb=/SingleMu/Run2012A-13Jul2012-v1/AOD=                                     &  190456 - 193621      & 0.796                \\ 
%\verb=/SingleMu/Run2012B-13Jul2012-v1/AOD=                                     &  193834 - 196531  & 4.412 \\ 
%\verb=/SingleMu/Run2012C-24Aug2012-v1/AOD=                                   &   198022 - 198523     & 0.473               \\ 
%\verb=/SingleMu/Run2012C-PromptReco-v2/AOD=                                  &   198934 - 203755    & 6.330                \\ 
%\verb=/SingleMu/Run2012C-EcalRecover_11Dec2012-v1/AOD=          & 201 191 - 201 191 & 0.133 \\
%\verb=/SingleMu/Run2012D-PromptReco-v1/AOD=                                  &  203768 - 208913   &  7.295 \\
%
%
%
% \hline\hline
%\end{tabular}
%\caption{\label{tab:DilDsets}Datasets and run ranges used in combination which contain signal events.}
%\end{center}
%\end{table}


\begin{table}[hbt]
\caption{\label{tab:DilDsets_lumi}Datasets and run ranges used in to measure the cross section. Taken in combination these datasets contain all of the events which may contribute to the measured signal.}
\begin{center}
\begin{tabular}{lc}\hline\hline
Name		 & Luminosity ($fb^{-1}$) \\ \hline
\verb=/DoubleMu/Run2012A-recover-06Aug2012-v1/AOD=                  &  0.081 \\ 
\verb=/DoubleMu/Run2012A-13Jul2012-v1/AOD=                                         & 0.796               \\ 
\verb=/DoubleMu/Run2012B-13Jul2012-v4/AOD=                                          & 4.412             \\ 
\verb=/DoubleMu/Run2012C-24Aug2012-v1/AOD=                                 & 0.473\\  
\verb=/DoubleMu/Run2012C-PromptReco-v2/AOD=                                    & 6.330                \\ 
\verb=/SingleMu/Run2012C-EcalRecover_11Dec2012-v1/AOD=           & 0.133\\
\verb=/DoubleMu/Run2012D-PromptReco-v1/AOD=                                 &  7.295 \\
%\verb=/DoubleMu/Run2012D-16Jan2013-v1/AOD=                                 &  207883 - 208307  \\

\verb=/DoubleElectron/Run2012A-recover-06Aug2012-v1/AOD=              & 0.081              \\ 
\verb=/DoubleElectron/Run2012A-13Jul2012-v1/AOD=                            & 0.796                    \\ 
\verb=/DoubleElectron/Run2012B-13Jul2012-v1/AOD=                           & 4.412\\ 
\verb=/DoubleElectron/Run2012C-24Aug2012-v1/AOD=                           & 0.473                 \\ 
\verb=/DoubleElectron/Run2012C-PromptReco-v2/AOD=                          & 6.330             \\ 
\verb=/SingleMu/Run2012C-EcalRecover_11Dec2012-v1/AOD=           & 0.133\\
\verb=/DoubleElectron/Run2012D-PromptReco-v1/AOD=                        &  7.295 \\
%\verb=/DoubleElectron/Run2012D-16Jan2013-v2/AOD=                        &   207883 - 208307 \\

\verb=/MuEG/Run2012A-recover-06Aug2012-v1/AOD=                              & 0.081            \\ 
\verb=/MuEG/Run2012A-13Jul2012-v1/AOD=                                                   & 0.796             \\ 
\verb=/MuEG/Run2012B-13Jul2012-v1/AOD=                                             & 4.412 \\ 
\verb=/MuEG/Run2012C-24Aug2012-v1/AOD=                                         & 0.473               \\ 
\verb=/MuEG/Run2012C-PromptReco-v2/AOD=                                         & 6.330              \\ 
\verb=/SingleMu/Run2012C-EcalRecover_11Dec2012-v1/AOD=           & 0.133\\
\verb=/MuEG/Run2012D-PromptReco-v1/AOD=                                      &  7.295 \\
%\verb=/MuEG/Run2012D-16Jan2013-v2/AOD=                                       &  203768 - 208307 \\

\verb=/SingleMu/Run2012A-recover-06Aug2012-v1/AOD=                              & 0.081          \\ 
\verb=/SingleMu/Run2012A-13Jul2012-v1/AOD=                                           & 0.796                \\ 
\verb=/SingleMu/Run2012B-13Jul2012-v1/AOD=                                       & 4.412 \\ 
\verb=/SingleMu/Run2012C-24Aug2012-v1/AOD=                                        & 0.473               \\ 
\verb=/SingleMu/Run2012C-PromptReco-v2/AOD=                                      & 6.330                \\ 
\verb=/SingleMu/Run2012C-EcalRecover_11Dec2012-v1/AOD=           & 0.133 \\
\verb=/SingleMu/Run2012D-PromptReco-v1/AOD=                                     &  7.295 \\



 \hline\hline
\end{tabular}

\end{center}
\end{table}








\begin{table}[hbt]
\caption{\label{tab:FRDsets}Datasets and run ranges used in combination to measure the lepton fake rates. These datasets contain the utility triggers that correspond to the signal triggers and ensure a discrete control region for the fake rate measurement.}
\begin{center}
\begin{tabular}{lc}\hline\hline
Name		& Run Range \\ \hline
\verb=/DoubleMu/Run2012A-recover-06Aug2012-v1/AOD=                 &   190782 - 190949\\ 
\verb=/DoubleMu/Run2012A-13Jul2012-v1/AOD=                                  &  190456 - 193621                     \\ 
\verb=/DoubleMu/Run2012B-13Jul2012-v4/AOD=                                  &  193834 - 196531                     \\ 
\verb=/DoubleMu/Run2012C-24Aug2012-v1/AOD=                                &  197770 - 198913 \\  
\verb=/DoubleMu/Run2012C-PromptReco-v2/AOD=                               &  198934 - 203755                     \\ 
\verb=/DoubleMu/Run2012D-PromptReco-v1/AOD=                               &  203768 - 208913  \\
\verb=/DoubleMu/Run2012D-16Jan2013-v1/AOD=                                 &  207883 - 208307  \\

\verb=/DoubleElectron/Run2012A-recover-06Aug2012-v1/AOD=         &   190782 - 190949                    \\ 
\verb=/DoubleElectron/Run2012A-13Jul2012-v1/AOD=                         & 190456 - 193621                      \\ 
\verb=/DoubleElectron/Run2012B-13Jul2012-v1/AOD=                         &  193834 - 196531 \\ 
\verb=/DoubleElectron/Run2012C-24Aug2012-v1/AOD=                       &   197770 - 198913                    \\ 
\verb=/DoubleElectron/Run2012C-PromptReco-v2/AOD=                     &    198934 - 203755                  \\ 
\verb=/DoubleElectron/Run2012D-PromptReco-v1/AOD=                      &  203768 - 208913  \\
\verb=/DoubleElectron/Run2012D-16Jan2013-v2/AOD=                        &   207883 - 208307 \\

\verb=/SingleMu/Run2012A-recover-06Aug2012-v1/AOD=                    &     190782 - 190949                  \\ 
\verb=/SingleMu/Run2012A-13Jul2012-v1/AOD=                                     &  190456 - 193621                     \\ 
\verb=/SingleMu/Run2012B-13Jul2012-v1/AOD=                                     &  193834 - 196531 \\ 
\verb=/SingleMu/Run2012C-24Aug2012-v1/AOD=                                   &   198022 - 198523                   \\ 
\verb=/SingleMu/Run2012C-PromptReco-v2/AOD=                                  &   198934 - 203755                    \\ 
\verb=/SingleMu/Run2012D-PromptReco-v1/AOD=                                  &  203768 - 208913   \\
 \hline\hline
\end{tabular}

\end{center}
\end{table}

%\subsection{Data Certification}
%\label{sec:data_details:cert}
%
%Only certified data from the datasets and run ranges listed in Appendix~\ref{sec:data_details:datasets} is included in the analysis.  The list of good data taking periods is taken from a combination of the following prompt and re-reco certifications:

%\clearpage	
	
	
	
	
\section{Monte Carlo Simulation Samples}
\label{sec:MCSamples}

All MC is produced by the Madgraph5~\cite{Alwall:2011uj} event generator, and interfaced with Pythia6~\cite{pythia6} for hadronization and showering. Finally a Geant4~\cite{geant4applications}~\cite{geant4toolkit} based model of the CMS detector is used to simulate particle interactions with the detector. The MC events and data are processed using the same reconstruction algorithms. Simulated events are scaled to the measured luminosity using highest order cross sections available at the time.

Monte Carlo samples are used for the prediction of rare SM tri-lepton yields as well as for studies of the background estimation methods.  
Background samples were produced with full simulation and reconstructed with a 53x CMSSW release as part of the Summer12\_DR53X campaign.
The $t\bar{t}Z$ signal sample has the same origin.    
All cross sections used in normalization are Next to Leading Order (NLO) when available.

Table~\ref{tab:IrreducibleMCSamples} contains a list of Monte Carlo samples contributing to the SM background from rare processes that is taken from simulation.  The cross section and equivalent luminosity of each sample is also provided. \ttZ \ is included in this table as it is used in the spillage subtraction in the fake rate. Table~\ref{tab:frEstimatedMCSamples} are used to help understand the behavior of the data derived fake rate. Additionally Table~\ref{tab:bEstimatedMCSamples} contains a list of Monte Carlo samples that are used for reference only to help gain insight into  the estimates from the method that predicts the contribution of events that have b-tags from radiation. Finally, more samples used for various studies may be found in Appendix~\ref{sec:mc_details}.\\

\begin{sidewaystable}[H]
\caption{\label{tab:IrreducibleMCSamples} MC datasets corresponding to contributions not covered by the data-driven methods.
Predicted yields from the SM samples listed here are used directly in the analysis. 
The common part of each dataset name Summer12\_DR53X-PU\_S10\_START53\_V7X-v1 is replaced with a shorthand Su12 V7X. 
All datasets are in the AODSIM data tier.}
\begin{center}
\begin{tabular}{lcc}
\hline\hline
Name														           & Cross section, pb & Luminosity, \fbinv \\ \hline
\verb=/TTZJets_8TeV-madgraph/Su12-v1=                                                                 & 0.206                     &       1021.68         \\ 
\verb=/TTWWJets_8TeV-madgraph/Su12 V7A=                                                            & 0.002037          &      106932       \\ 
\verb=/TTWJets_8TeV-madgraph/Su12 V7A=                                                               & 0.232             &        845.026         \\ 
\verb=/TTGJets_8TeV-madgraph/Su12 V19=                                                                & 2.166             &       775.602          \\ 
\verb=/TBZToLL_4F_TuneZ2star_8TeV-madgraph-tauola/Su 12 V7C=                 &  0.0114             &         13026.7        \\
\verb=/WZZNoGstarJets_8TeV-madgraph/Su12 V7A=                                               & 0.01922           &       12946.7        \\ 
\verb=/ZZZNoGstarJets_8TeV-madgraph/Su12 V7A=                                                 & 0.004587          &     40692.6          \\ 
\hline\hline
\end{tabular}

\end{center}
\end{sidewaystable}



\begin{sidewaystable}[H]
\caption{\label{tab:frEstimatedMCSamples} MC datasets that do not contribute to MC Pred.  The contribution to the background from these processes is covered by data-driven methods, but expected yields based on simulation are nevertheless provided as a reference.
Predicted yields from the SM samples listed here are used directly in the analysis. 
The common part of each dataset name Summer12\_DR53X-PU\_S10\_START53\_V7A is replaced with a shorthand Su12 V7A. 
All datasets are in the AODSIM data tier.}
\begin{center}
\begin{tabular}{lcc}
\hline\hline
Name                                                                                                                             & Cross section, pb & Luminosity, \fbinv \\ \hline
\verb=/TTJets_SemiLeptMGDecays_8TeV-madgraph-tauola/Su 12 V7C-v1= & 102.50 & 247.657\\
\verb=/TTJets_FullLeptMGDecays_8TeV-madgraph/S 12V7A-v2=                    &  24.56       & 493.445\\
\verb=/DY1JetsToLL_M-50_TuneZ2Star_8TeV-madgraph/Su 12 V7A=           &   671.83          &       35.4483 \\
\verb=/DY2JetsToLL_M-50_TuneZ2Star_8TeV-madgraph/Su 12 V7C=           &  216.76            &    100.444    \\
\verb=/DY3JetsToLL_M-50_TuneZ2Star_8TeV-madgraph/Su 12 V7A=           &   61.2          &      178.847  \\
\verb=/DY4JetsToLL_M-50_TuneZ2Star_8TeV-madgraph/Su 12 V7A=           &  27.59           &   232.071     \\ 
\verb=/W1JetsToLNu_TuneZ2Star_8TeV-madgraph/Su 12 V7A=                     &    6663                    &  3.47315 \\
\verb=/W2JetsToLNu_TuneZ2Star_8TeV-madgraph/Su 12 V7A=                     &    2159                    &  15.7688 \\
\verb=/W3JetsToLNu_TuneZ2Star_8TeV-madgraph/Su 12 V7A=                     &   640                    &  24.2805 \\
\verb=/W4JetsToLNu_TuneZ2Star_8TeV-madgraph/Su 12 V7A=                     &    264                    & 50.6924 \\
\verb=/WWJetsTo2L2Nu_TuneZ2star_8TeV-madgraph-tauola/Su 12 V7A=      &  5.8123            &     332.611             \\ 
\verb=/WZJetsTo2L2Q_TuneZ2star_8TeV-madgraph-tauola/Su 12 V7A=      &    2.206            &        1457.84          \\
\verb=/ZZJetsTo2L2Q_TuneZ2star_8TeV-madgraph-tauola/Su 12 V7A=      &     2.4487          &        790.921          \\
\hline\hline
\end{tabular}

\end{center}
\end{sidewaystable}




%\begin{sidewaystable}[H]
%\begin{center}
%\begin{tabular}{l}
%\hline\hline
%Name                                                                                                                              \\ \hline
%\verb=/QCD_Pt_20_MuEnrichedPt_15_TuneZ2star_8TeV_pythia6/Su 12 V7A-v3=                       \\
%\verb=/QCD_Pt-15to20_MuEnrichedPt5_TuneZ2star_8TeV_pythia6/Su 12 V7A-v2=                       \\
%\verb=/QCD_Pt-20to30_MuEnrichedPt5_TuneZ2star_8TeV_pythia6/Su 12 V7A-v1=                        \\
%\verb=/QCD_Pt-30to50_MuEnrichedPt5_TuneZ2star_8TeV_pythia6/Su 12 V7A-v1=                        \\
%\verb=/QCD_Pt-50to80_MuEnrichedPt5_TuneZ2star_8TeV_pythia6/Su 12 V7A-v1=                     \\
%\verb=/QCD_Pt-80to120_MuEnrichedPt5_TuneZ2star_8TeV_pythia6/Su 12 V7A-v1=                       \\
%\verb=/QCD_Pt-120to170_MuEnrichedPt5_TuneZ2star_8TeV_pythia6/Su 12 V7A-v1=                       \\
%\verb=/QCD_Pt-5to15_TuneZ2star_8TeV_pythia6/Su 12 V7A-v1=                        \\
%\verb=/QCD_Pt-15to30_TuneZ2star_8TeV_pythia6/Su 12 V7A-v2=                      \\
%\verb=/QCD_Pt-30to50_TuneZ2star_8TeV_pythia6/Su 12 V7A-v2=                      \\
%\verb=/QCD_Pt-50to80_TuneZ2star_8TeV_pythia6/Su 12 V7A-v2=                        \\
%\verb=/QCD_Pt-80to120_TuneZ2star_8TeV_pythia6/Su 12 V7A-v3=                     \\
%\verb=/QCD_Pt-120to170_TuneZ2star_8TeV_pythia6/Su 12 V7A-v3=                       \\
%\verb=/QCD_Pt-170to300_TuneZ2star_8TeV_pythia6/Su 12 V7A-v2=                        \\
%\hline\hline
%\end{tabular}
%\caption{\label{tab:frQCDMCSamples} MC datasets that are used in calculated the Fake Rate used in the closure tests for the purpose of determining the systematic uncertainty on the method as well as correcting the central value of the prediction. 
%The common part of each dataset name {\tt Summer12\_DR53X-PU\_S10\_START53\_V7A} is replaced with a shorthand {\tt Su12 V7A}. 
%All datasets are in the AODSIM data tier.}
%\end{center}
%\end{sidewaystable}
%\clearpage




\begin{sidewaystable}[H]
\caption{\label{tab:bEstimatedMCSamples} MC datasets that do not contribute to MC Pred.  The contribution to the background from these processes is covered by data-driven methods (b-tag estimation from radian jets), but expected yields based on simulation are nevertheless provided as a reference.
Predicted yields from the SM samples listed here are used directly in the analysis. 
The common part of each dataset name Summer12\textunderscore DR53X-PU\textunderscore S10\textunderscore START53\textunderscore V7A is replaced with a shorthand Su12 V7A. 
All datasets are in the AODSIM data tier.}
\begin{center}
\begin{tabular}{lcc}
\hline\hline
Name                                                                                                                             & Cross section, pb & Luminosity, \fbinv \\ \hline
\verb=/WZJetsTo3LNu_TuneZ2_8TeV-madgraph-tauola/Su 12 V7A=      &       1.0575        &        1908.25        \\
\verb=/ZZJetsTo4L_TuneZ2star_8TeV-madgraph-tauola/Su 12 V7A=      &         0.176908      &         27177.4         \\
\verb=/WWGJets_8TeV-madgraph/Su12 V7A=                                                            & 0.528             &        407.426         \\ 
\verb=/WWWJets_8TeV-madgraph/Su12 V7A=                                                            & 0.08217           &     2737.02           \\ 
\verb=/WWZNoGstarJets_8TeV-madgraph/Su12 V7A=                                              & 0.0633            &        3832.94        \\ 
\hline\hline
\end{tabular}

\end{center}
\end{sidewaystable}

\clearpage


%\begin{sidewaystable}[H]
%\begin{center}
%\begin{tabular}{lcc}
%\hline\hline
%Name                                                                                                                             & Cross section, pb & Luminosity, \fbinv \\ \hline
%\verb=/DYJetsToLL_M-50_TuneZ2Star_8TeV-madgraph-tarball/Su12-v1=    &        3532.8149     &         8.62188         \\ 
%\verb=/WZJetsTo3LNu_TuneZ2_8TeV-madgraph-tauola/Su 12 V7A=      &        1.0575       &              1908.25    \\
%\hline\hline
%\end{tabular}
%\caption{\label{tab:bRateComparison} MC datasets used to validate the b-tag content in a di-lepton and tril-lepton sample against each other as a demonstration of the validity of the method used to predict contribution to the background via b-tags that come from radiation. Note that the DY sample differs from that used in the rest of the analysis. This one was specifically chosen because it treats the b-quark's mass the same as in the used WZ sample.
%The common part of each dataset name {\tt Summer12\_DR53X-PU\_S10\_START53\_V7A} is replaced with a shorthand {\tt Su12 V7A}. 
%All datasets are in the AODSIM data tier.}
%\end{center}
%\end{sidewaystable}
%
%
%\begin{sidewaystable}[H]
%\begin{center}
%\begin{tabular}{lcc}
%\hline\hline
%Name                                                                                                                             & Cross section, pb & Luminosity, \fbinv \\ \hline
%\verb=/TTZJets_8TeV-madgraph_v2/Su 12 V7A=                                         &        0.2057     &        1021.68          \\ 
%\verb=/ttbarZ_8TeV-Madspin_aMCatNLO-herwig/Su 12 V19=                    &        0.2057       &                  \\
%\hline\hline
%\end{tabular}
%\caption{\label{tab:ttZGeneratorMCs} MC datasets used to determine the signal systematic uncertainty due to the type of generator used. They are both normalized to the same cross section. Madgraph is an LO generator while aMC@NLO is an NLO generator. The madgraph sample is the same as the one listed in table~\ref{tab:IrreducibleMCSamples}.
%The common part of each dataset name {\tt Summer12\_DR53X-PU\_S10\_START53\_V7A} is replaced with a shorthand {\tt Su12 V7A}. 
%All datasets are in the AODSIM data tier.}
%\end{center}
%\end{sidewaystable}
%
%\clearpage
