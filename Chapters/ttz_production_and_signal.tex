	\section{specifics of ttZ production}
	\section{specifics of ttZ decay}
	After the hard collision, 3 objects exist that will be indirectly studied: a top, an anti-top, and a Z boson. For all intents and purposes, the top and anti-top particles behave similarly just swapping the matter and anti-matter labels in the decay products. A top quark is very massive at \~ 173 \GeV of energy and thus decays very quickly. There is no time for it to decay via the strong force and a hadronize, so it does not form jets. The top quark can only decay to a W boson plus a down-type quark which is a b-quark the vast majority of the time. \\
	
	The b-quarks from the top and anti-top will hadronize and form b-jets which can be reconstructed with varying degrees of efficiency and purity (see Section ~\ref{}) and act as markers for top events as high quality b-tags are fairly except in top or boson decays. The W and Z bosons will decay either to hadronic or leptonic final states. The W decays to either a lepton and a neutrino (\~ 11\% per lepton flavor ~\cite{PDG}) or to an up-type quark and a down-type quark (weakly). For the Z boson, quark anti-quark pairs are common (the most being \bbbar pairs which decay contributes to the \ttZ backgrounds) and also to lepton anti-lepton pairs (\~ 3.3\% per lepton flavor ~\cite{PDG}).\\
	
	Thus several final states exist:
	\begin{enumerate}
	\item 8 quarks (at least 2 of which are b-quarks)
	\item 6 quarks (at least 2 of which are b-quarks), 1 lepton, and 1 neutrino
	\item 6 quarks (at least 2 of which are b-quarks) and 2 leptons
	\item 4 quarks (at least 2 of which are b-quarks), 2 leptons, and 2 neutrinos
	\item 4 quarks (at least 2 of which are b-quarks), 3 leptons, and 1 neutrino
	\item 2 b-quarks, 4 leptons, and 2 neutrinos
	\end{enumerate}
	
	Thus there are several different signatures to look for in order to identify a \ttZ decay. As shown in Fig. ~\ref{fig:ttZ_decay_rates}, hadronic decays of the bosons are more common. However, the rate of decay must be balanced agains how unique and easy to distinguish (measure and reconstruct its elements in the detector) from the background processes.
	
	\begin{figure}[h]
\begin{center}
\includegraphics[width=0.48\linewidth]{Figs/placeholder.pdf}
\caption{\label{fig:ttZ_decay_rates}
Fraction of time \ttZ decays to various final states.
}
\end{center}
\end{figure} 

	
	\section{reasons for choosing 3 lepton final state}