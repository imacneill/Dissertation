\documentclass[12pt,chapterheads,oneside]{ucsd}

\usepackage{amsmath, amscd, amssymb, amsthm}
\usepackage{graphicx}
\usepackage{xfrac}
\usepackage{color}
\usepackage{multirow}
\usepackage{multicol}
\usepackage{ifthen}
\usepackage{xspace}
\usepackage{calc}
%\usepackage{slashbox}
\usepackage{subfig}
\usepackage[T1]{fontenc}
\usepackage{mathptmx}
\usepackage{makeidx}
\usepackage[bottom]{footmisc}
\usepackage[hyphens]{url}
\usepackage[color=red!40,textwidth=24mm,textsize=footnotesize]{todonotes}
\usepackage[hidelinks,linktocpage,breaklinks]{hyperref}                                  
\usepackage{rotating}
\usepackage{afterpage}
\usepackage{xparse}
\usepackage{lineno}
\usepackage{slashed}
\usepackage{bm}

\hypersetup{ pdfauthor   = {Ian MacNeill},
             pdftitle    = {Measurement of top quark-antiquark pair production in association with a Z boson with a trilepton final state in pp collisions at \sqrt{s} = 8 TeV},
             pdfkeywords = {LHC CERN CMS top Standard Model SM Ian MacNeill},
             pdfcreator  = {LaTeX with hyperref package},
             pdfproducer = {LaTeX} }

\input{include/definitions}                                                        
\includeonly{include/frontmatter}                                                    

\setlength{\parindent}{0.5in}
\setcounter{secnumdepth}{2}
\setcounter{tocdepth}{2}

\makeindex
\synctex=1

\hyphenation{back-ground-only}

\begin{document}
\graphicspath{
{Figs/}
%{intro/figs/}
%{cms/figs/}
%{ss/figs/}
%{bkgd/figs/}
%{eff/figs/}
%{results/figs/}
%{results/yields/high_pt/exclusive}
%{results/yields/low_pt/exclusive}
}

% No symbols, formulas, superscripts, or Greek letters are allowed
% in your title.
\title{Measurement of top quark-antiquark pair production in association with a Z boson with a trilepton final state in pp collisions at 8 TeV center of mass energy}

\author{Ian Christopher MacNeill}
\degreeyear{2015}

% Master's Degree theses will NOT be formatted properly with this file.
\degreetitle{Doctor of Philosophy} 

\field{Physics}
\chair{Professor Avraham Yagil}

%  The rest of the committee members  must be alphabetized by last name.
\othermembers{
Professor Claudio Campagnari\\ 
Professor Aneesh Manohar\\
Professor George Tynan\\
Professor Frank W\"urthwein\\
}
\numberofmembers{5} % |chair| + |cochair| + |othermembers|


\begin{frontmatter}
\makefrontmatter                                                               

%% ----------------------------------------------------------------------- %%
%% DEDICATION
%% ----------------------------------------------------------------------- %%

\begin{mydedication}    
        	\vfil                                                 
	\noindent {\it To my parents,} \hfill \\
	\hfill {\it who started me asking questions and taught me to value knowledge.} \\
	\vfil
	\vfil
	\vfil
\end{mydedication}                                                               
\clearpage 

%% ----------------------------------------------------------------------- %%
%% EPIGRAPH
%% ----------------------------------------------------------------------- %%

%  The same choices that applied to the dedication apply here.
% \begin{epigraph} % The style file will position the text for you.              
%   \it{Mon seul d\'esir est de m'enrichir de nouvelles pens\'ees exaltantes.} \\
%   ---Ren\'e Magritte
% \end{epigraph}                                                                 
\begin{myepigraph} % You position the text yourself.                           
  \vfil                                                                        
  \vfil 
  \hfill {\it Numquam aliud natura, aliud sapientia dicit.} \\
  \vfil 
  \noindent {\it Never does nature say one thing and wisdom say another.} \hfill \\
  \vfil 
  \hfill ---Juvenal
  \vfil 
\end{myepigraph}                                                               

\tableofcontents
\listoffigures  % Uncomment if you have any Figures                            
\listoftables   % Uncomment if you have any Tables                             

\begin{acknowledgements}                                                       
Well\ldots it's finally here. This work is finished, simultaneously seeming to pass slowly and at the same time taking me entirely by surprise that it's done. I have to give much of the credit for the completion of this work to my advisor, Avi Yagil, and post doc task master, Slava Krutelyov. They helped immensely to raise the quality of this work and the quality of this Ph.D. student. Other collaborators deserve to be thanked. Each provided their own lessons, outlooks, and way of doing things. If I did not learn from them, it is certainly not their fault. Frank W\"urthwein, Claudio Campagnari, Frank Golf, Giuseppe Cerati, Dave Evans, Ben Hooberman, Verena Martinez, Ryan Kelley, Vince Welke, and Warren Andrews.  I'd like to point out a special thanks to Frank and Warren for getting me started, and Ryan for being a sounding board and technical expert near the end of the work. Finally, Boris Mangano, Amanda Deisher, Lukas B\"ani, and Matthew Walker were excellent collaborators on the combined cross section and extremely patient dealing with someone not working at CERN. \\

I suppose I should blame my parents (Fletcher and Martha), or at least thank them, for starting me on this road. They both encouraged me to ask questions and instilled a sense of wonder and exploration in me directed towards the world we live in. I learned to love reading, sports, building things with my hands, and figuring out how things work because of their influence. To be honest, there's plenty of blame to go around here. My grandparents, none of whom lived to see me finish this work, started early with teaching me how to make plant clippings, play scrabble and chess, and to love cooking. Yes, many of the things listed don't seem related, but being a physicist is about more than just knowing math and sitting in a windowless office. It comes down to wanting to understand how something works and continuing to ask questions until you do. All of them deserve a great deal of thanks for teaching me the rewards of striving to understand.\\

For the past few years, I have had some wonderful companions who have figuratively and, unfortunately, literally propped me up. My partner, Wyn, deserves a lot of credit for loading all of our possessions into a truck and blindly driving across the country to a strange new land, California, with me. We've both sacrificed out here by being nearly 3000 miles away from our families, but have also made a very rewarding life for ourselves. Wyn has been unflaggingly encouraging and supportive. Wyn also helped to usher in two extra companions, Freddie and Fiona. As I write, Freddie is glued to my side. He's either encouraging me to write, or making sure I don't forget to feed him. It's hard to say which. Fiona was a cranky bitch (literally), but she also knew just the right way to show love and attention before she passed away. Until that point, she helped keep me company while I worked late and helped to relieve stress and provide an outlet for my energy other than physics. Despite her insistence to the contrary, she was a terrible coder. Freddie still endeavors to fill this role.\\

There are many other people who have helped me while I've lived out here and worked on this Ph.D. They include my aunt, Susan, who took us in the first night we arrived and has been a helpful and a familiar face ever since then. Credit goes to all of my friends on the cycling team, who have given me something to do other than sit in a windowless office. The beer we've had together and the bikes we've ridden (\ldots and crashed) together have kept me sane and balanced these past few years. Finally, I'd like to thank the rest of the graduate students at UCSD. You have been a great group of people to learn with.\\

\end{acknowledgements}                                                         

\begin{vitapage}                                                               
\begin{vita}                                                                   
  \item[2009] B.~A. in Physics, University of Pennsylvania
  \item[2011] M.~S. in Physics, University of California, San Diego
  \item[2015] Ph.~D. in Physics, University of California, San Diego       
\end{vita}                                                                     
\begin{publications}                                                           
 \item Measurement of top quark-antiquark pair production in association with a W or Z boson in pp collisions at $\sqrt{s}$ = 8 TeV, {\it CMS Collaboration}, The European Physical Journal C 74 (2014) No 9, doi:10.1140/epjc/s10052-014-3060-7, [arXiv:1406.7830 [hep-ex]]


 
 %\item Search for new physics in events with same-sign dileptons and b jets in pp collisions at $\sqrt{s} = 8$ TeV, {\it CMS Collaboration}, JHEP 1303 (2013) 037 [arXiv:1212.6194 [hep-ex]] % HCP
  %\item 2013!!!Search for new physics in events with same-sign dileptons and b jets in pp collisions at $\sqrt{s} = 8$ TeV, {\it CMS Collaboration}, JHEP 1303 (2013) 037 [arXiv:1212.6194 [hep-ex]] % 2013
\end{publications}                                                             
\end{vitapage}                                                                 
                                                                               

%% ABSTRACT
%  Doctoral dissertation abstracts should not exceed 350 words. 
%   The abstract may continue to a second page if necessary.
\begin{abstract}
A search for Z boson production with associated top-antitop pairs is performed with a final state which contains trileptons and b-tagged jets. The measurement is performed on the complete dataset of pp collisions at a center-of-mass energy of 8 TeV collected at the CMS detector, for a total of \lumi = 19.5~\fbinv. A cross section is calculated for this final state and compared to the standard model prediction. The cross section for \ttZ production is measured to be $\sigma=194 _{-89} ^{+105}$ \ fb. The measured cross section to theoretical cross section ratio is $0.94 _{-0.43} ^{+0.51}$. The process is measured with a significance of 2.33. This measurement is then combined with an outside 4 lepton channel measurement for an equivalent cross section measurement of $\sigma=200 _{-76} ^{+90} $ \ fb and a significance of 3.1.
 

\end{abstract}


\end{frontmatter}

% \linenumbers


\chapter{Introduction}     
	(I would let this double as a theory section since there is significantly less theory involved in a standard model measurement than say a susy search).
	\section{Standard model}
		(focus on bosons and tops and leptons)
	\section{pp collisions}          
		(focus on proton collisions creating tops and bosons)
	\section{Decays}
    		(focus on boson decays to quarks and leptons to help motivate the signature later)

\chapter{Detector Description}
	\section{Obligatory mention of LHC and various experiments}
	\section{Description of CMS detector}        
		(focus on parts and descriptions that help with e and mu id, isolation measurements, b-tagging, and jet measurements to motivate the reconstruction description of these later on)
	\section{Luminosity and triggering}        
		(both will be somewhat short)

\chapter{Particle Reconstruction}
                (the order may need to be changed around in the subsections)
	\section{charged particle reconstruction}
	\section{vertex reconstruction}
	\section{particle flow}
	\section{electrons}
	\section{muons}
	\section{jets}
	\section{b-tags}

\chapter{ttZ and Backgrounds}
	\section{specifics of ttZ production}
	\section{specifics of ttZ decay}
	\section{reasons for choosing 3 lepton final state}
	\section{Backgrounds}
		\subsection{explain a fake lepton, explain a mistag, explain other sources of b-jets, explain other sources of jets}
        		\subsection{describe the irreducible backgrounds} 
			(ttW, ttWW, ttG, etc)
        		\subsection{describe the backgrounds with fake leptons} 
			(ttbar, WW, etc)
        		\subsection{describe the backgrounds with b-tags not from a top decay} 
			(WZ, ZZ, etc)

(Alternatively, I can group the backgrounds by process instead of by how they become backgrounds)
	\section{Top backgrounds} (ttbar, ttW, ttG, etc)
	\section{Mono-boson backgrounds} (W, Z)
	\section{Di-boson backgrounds} (WW, WZ, etc)
	\section{Tri-boson backgrounds} (WWW, WWZ, etc)

\chapter{Sources of Real and Simulated Collsions}
\label{ch:samples}
\section{Collision Data Samples}
	
	This document presents results obtained from data of proton-proton collisions at 8 TeV center of mass energy. A total luminosity of \intLumi \ is used, which corresponds to the totality of data to be measured at this center of mass energy at the LHC. The data was collected by the CMS detector up through the end of 2012. This analysis relies on a mixture of data driven methods and MC simulation to estimate backgrounds. Additionally, signal events are generated for use in this analysis. 
	
	
	Data used is a combination of prompt (Run2012C v2 and D) and re-reco (Run2012A, B, part of C, and part of D) data.  
Run2012A comes from both the 13July re-reco and 06August recovery campaigns while Run2012B is only from the former.  
Data from Run2012C v1, corresponding to ~0.5 \fbinv, comes from the 24Aug re-reco while v2 is from prompt. Rounding out Run2012C is an addition Dec11 Ecal Recovery rereco for 0.133 \fbinv. Additionally, data for Run2012D comes from prompt.
    
Only events from certified data-taking periods are used.  
The selection of good run and luminosity sections comes from a combination of prompt and re-reco certification.  
The certified dataset considered in this document covers runs up to 208686 inclusive and corresponds to \intLumi \ and an associated error of 2.6\%~\cite{lumi12up}. 


	
	
	
Signal events are selected from the datasets and run ranges listed in Table~\ref{tab:DilDsets}.  The fake rate measurement is performed using those listed in Table~\ref{tab:FRDsets}.  Runs found in the Run2012A 06Aug2012 recover are excluded from the Run2012A 13Jul2012 re-reco dataset. 

\begin{table}[hbt]
\begin{center}
\begin{tabular}{lcc}\hline\hline
Name		& Run Range & Luminosity ($fb^{-1}$) \\ \hline
\verb=/DoubleMu/Run2012A-recover-06Aug2012-v1/AOD=                 & 190782 - 190949 &  0.081 \\ 
\verb=/DoubleMu/Run2012A-13Jul2012-v1/AOD=                                  &  190456 - 193621       & 0.796               \\ 
\verb=/DoubleMu/Run2012B-13Jul2012-v4/AOD=                                  &  193834 - 196531        & 4.412             \\ 
\verb=/DoubleMu/Run2012C-24Aug2012-v1/AOD=                                &  197770 - 198913  & 0.473\\  
\verb=/DoubleMu/Run2012C-PromptReco-v2/AOD=                               &  198934 - 203755     & 6.330                \\ 
\verb=/SingleMu/Run2012C-EcalRecover_11Dec2012-v1/AOD=          & 201 191 - 201 191 & 0.133\\
\verb=/DoubleMu/Run2012D-PromptReco-v1/AOD=                               &  203768 - 208913  &  7.295 \\
%\verb=/DoubleMu/Run2012D-16Jan2013-v1/AOD=                                 &  207883 - 208307  \\

\verb=/DoubleElectron/Run2012A-recover-06Aug2012-v1/AOD=         &    190782 - 190949     & 0.081              \\ 
\verb=/DoubleElectron/Run2012A-13Jul2012-v1/AOD=                         & 190456 - 193621   & 0.796                    \\ 
\verb=/DoubleElectron/Run2012B-13Jul2012-v1/AOD=                         &  193834 - 196531  & 4.412\\ 
\verb=/DoubleElectron/Run2012C-24Aug2012-v1/AOD=                       &  197770 - 198913    & 0.473                 \\ 
\verb=/DoubleElectron/Run2012C-PromptReco-v2/AOD=                     &   198934 - 203755     & 6.330             \\ 
\verb=/SingleMu/Run2012C-EcalRecover_11Dec2012-v1/AOD=          & 201 191 - 201 191 & 0.133\\
\verb=/DoubleElectron/Run2012D-PromptReco-v1/AOD=                      &  203768 - 208913  &  7.295 \\
%\verb=/DoubleElectron/Run2012D-16Jan2013-v2/AOD=                        &   207883 - 208307 \\

\verb=/MuEG/Run2012A-recover-06Aug2012-v1/AOD=                          &      190782 - 190949     & 0.081            \\ 
\verb=/MuEG/Run2012A-13Jul2012-v1/AOD=                                          &  190456 -193621         & 0.796             \\ 
\verb=/MuEG/Run2012B-13Jul2012-v1/AOD=                                         &  193834 -196531      & 4.412 \\ 
\verb=/MuEG/Run2012C-24Aug2012-v1/AOD=                                      &   197770 - 198913     & 0.473               \\ 
\verb=/MuEG/Run2012C-PromptReco-v2/AOD=                                     &   198934 - 203755      & 6.330              \\ 
\verb=/SingleMu/Run2012C-EcalRecover_11Dec2012-v1/AOD=          & 201 191 - 201 191 & 0.133\\
\verb=/MuEG/Run2012D-PromptReco-v1/AOD=                                     &  203768 - 208913  &  7.295 \\
%\verb=/MuEG/Run2012D-16Jan2013-v2/AOD=                                       &  203768 - 208307 \\

\verb=/SingleMu/Run2012A-recover-06Aug2012-v1/AOD=                    &   190782 - 190949          & 0.081          \\ 
\verb=/SingleMu/Run2012A-13Jul2012-v1/AOD=                                     &  190456 - 193621      & 0.796                \\ 
\verb=/SingleMu/Run2012B-13Jul2012-v1/AOD=                                     &  193834 - 196531  & 4.412 \\ 
\verb=/SingleMu/Run2012C-24Aug2012-v1/AOD=                                   &   198022 - 198523     & 0.473               \\ 
\verb=/SingleMu/Run2012C-PromptReco-v2/AOD=                                  &   198934 - 203755    & 6.330                \\ 
\verb=/SingleMu/Run2012C-EcalRecover_11Dec2012-v1/AOD=          & 201 191 - 201 191 & 0.133 \\
\verb=/SingleMu/Run2012D-PromptReco-v1/AOD=                                  &  203768 - 208913   &  7.295 \\



 \hline\hline
\end{tabular}
\caption{\label{tab:DilDsets}Datasets and run ranges used in combination which contain signal events.}
\end{center}
\end{table}

\begin{table}[hbt]
\begin{center}
\begin{tabular}{lc}\hline\hline
Name		& Run Range \\ \hline
\verb=/DoubleMu/Run2012A-recover-06Aug2012-v1/AOD=                 &   190782 - 190949\\ 
\verb=/DoubleMu/Run2012A-13Jul2012-v1/AOD=                                  &  190456 - 193621                     \\ 
\verb=/DoubleMu/Run2012B-13Jul2012-v4/AOD=                                  &  193834 - 196531                     \\ 
\verb=/DoubleMu/Run2012C-24Aug2012-v1/AOD=                                &  197770 - 198913 \\  
\verb=/DoubleMu/Run2012C-PromptReco-v2/AOD=                               &  198934 - 203755                     \\ 
\verb=/DoubleMu/Run2012D-PromptReco-v1/AOD=                               &  203768 - 208913  \\
\verb=/DoubleMu/Run2012D-16Jan2013-v1/AOD=                                 &  207883 - 208307  \\

\verb=/DoubleElectron/Run2012A-recover-06Aug2012-v1/AOD=         &   190782 - 190949                    \\ 
\verb=/DoubleElectron/Run2012A-13Jul2012-v1/AOD=                         & 190456 - 193621                      \\ 
\verb=/DoubleElectron/Run2012B-13Jul2012-v1/AOD=                         &  193834 - 196531 \\ 
\verb=/DoubleElectron/Run2012C-24Aug2012-v1/AOD=                       &   197770 - 198913                    \\ 
\verb=/DoubleElectron/Run2012C-PromptReco-v2/AOD=                     &    198934 - 203755                  \\ 
\verb=/DoubleElectron/Run2012D-PromptReco-v1/AOD=                      &  203768 - 208913  \\
\verb=/DoubleElectron/Run2012D-16Jan2013-v2/AOD=                        &   207883 - 208307 \\

\verb=/SingleMu/Run2012A-recover-06Aug2012-v1/AOD=                    &     190782 - 190949                  \\ 
\verb=/SingleMu/Run2012A-13Jul2012-v1/AOD=                                     &  190456 - 193621                     \\ 
\verb=/SingleMu/Run2012B-13Jul2012-v1/AOD=                                     &  193834 - 196531 \\ 
\verb=/SingleMu/Run2012C-24Aug2012-v1/AOD=                                   &   198022 - 198523                   \\ 
\verb=/SingleMu/Run2012C-PromptReco-v2/AOD=                                  &   198934 - 203755                    \\ 
\verb=/SingleMu/Run2012D-PromptReco-v1/AOD=                                  &  203768 - 208913   \\
 \hline\hline
\end{tabular}
\caption{\label{tab:FRDsets}Datasets and run ranges used in combination to measure the lepton fake rates.}
\end{center}
\end{table}

\subsection{Data Certification}
\label{sec:data_details:cert}

Only certified data from the datasets and run ranges listed in Appendix~\ref{sec:data_details:datasets} is included in the analysis.  The list of good data taking periods is taken from a combination of the following prompt and re-reco certifications:


\clearpage	
	
	
	
	
\section{Monte Carlo Simulation Samples}
\label{sec:MCSamples}

All MC is produced by the Madgraph5~\cite{Alwall:2011uj} event generator, and interfaced with Pythia6~\cite{pythia6} for hadronization and showering. Finally a Geant4~\cite{geant4applications}~\cite{geant4toolkit} based model of the CMS detector is used to simulate particle interactions with the detector. The MC events and data are processed using the same reconstruction algorithms. Simulated events are scaled to the measured luminosity using highest order cross sections available at the time.

Monte Carlo samples are used for the prediction of rare SM tri-lepton yields as well as for studies of the background estimation methods.  
Background samples were produced with full simulation and reconstructed with a 53x CMSSW release as part of the Summer12\_DR53X campaign.
The $t\bar{t}Z$ signal sample has the same origin.    
All cross sections used in normalization are Next to Leading Order (NLO) when available.

Table~\ref{tab:IrreducibleMCSamples} contains a list of Monte Carlo samples contributing to the SM background from rare processes that is taken from simulation.  The cross section and equivalent luminosity of each sample is also provided. \ttZ \ is included in this table as it is used in the spillage subtraction in the fake rate.

\begin{sidewaystable}[H]
\begin{center}
\begin{tabular}{lcc}
\hline\hline
Name														           & Cross section, pb & Luminosity, \fbinv \\ \hline
\verb=/TTZJets_8TeV-madgraph/Su12-v1=                                                                 & 0.206                     &       1021.68         \\ 
\verb=/TTWWJets_8TeV-madgraph/Su12 V7A=                                                            & 0.002037          &      106932       \\ 
\verb=/TTWJets_8TeV-madgraph/Su12 V7A=                                                               & 0.232             &        845.026         \\ 
\verb=/TTGJets_8TeV-madgraph/Su12 V19=                                                                & 2.166             &       775.602          \\ 
\verb=/TBZToLL_4F_TuneZ2star_8TeV-madgraph-tauola/Su 12 V7C=                 &  0.0114             &         13026.7        \\
\verb=/WZZNoGstarJets_8TeV-madgraph/Su12 V7A=                                               & 0.01922           &       12946.7        \\ 
\verb=/ZZZNoGstarJets_8TeV-madgraph/Su12 V7A=                                                 & 0.004587          &     40692.6          \\ 
\hline\hline
\end{tabular}
\caption{\label{tab:IrreducibleMCSamples} MC datasets corresponding to contributions not covered by the data-driven methods.
Predicted yields from the SM samples listed here are used directly in the analysis. 
The common part of each dataset name {\tt Summer12\_DR53X-PU\_S10\_START53\_V7X-v1} is replaced with a shorthand {\tt Su12 V7X}. 
All datasets are in the AODSIM data tier.}
\end{center}
\end{sidewaystable}

\clearpage

\begin{sidewaystable}[H]
\begin{center}
\begin{tabular}{lcc}
\hline\hline
Name                                                                                                                             & Cross section, pb & Luminosity, \fbinv \\ \hline
\verb=/TTJets_SemiLeptMGDecays_8TeV-madgraph-tauola/Su 12 V7C-v1= & 102.50 & 247.657\\
\verb=/TTJets_FullLeptMGDecays_8TeV-madgraph/S 12V7A-v2=                    &  24.56       & 493.445\\
\verb=/DY1JetsToLL_M-50_TuneZ2Star_8TeV-madgraph/Su 12 V7A=           &   671.83          &       35.4483 \\
\verb=/DY2JetsToLL_M-50_TuneZ2Star_8TeV-madgraph/Su 12 V7C=           &  216.76            &    100.444    \\
\verb=/DY3JetsToLL_M-50_TuneZ2Star_8TeV-madgraph/Su 12 V7A=           &   61.2          &      178.847  \\
\verb=/DY4JetsToLL_M-50_TuneZ2Star_8TeV-madgraph/Su 12 V7A=           &  27.59           &   232.071     \\ 
\verb=/W1JetsToLNu_TuneZ2Star_8TeV-madgraph/Su 12 V7A=                     &    6663                    &  3.47315 \\
\verb=/W2JetsToLNu_TuneZ2Star_8TeV-madgraph/Su 12 V7A=                     &    2159                    &  15.7688 \\
\verb=/W3JetsToLNu_TuneZ2Star_8TeV-madgraph/Su 12 V7A=                     &   640                    &  24.2805 \\
\verb=/W4JetsToLNu_TuneZ2Star_8TeV-madgraph/Su 12 V7A=                     &    264                    & 50.6924 \\
\verb=/WWJetsTo2L2Nu_TuneZ2star_8TeV-madgraph-tauola/Su 12 V7A=      &  5.8123            &     332.611             \\ 
\verb=/WZJetsTo2L2Q_TuneZ2star_8TeV-madgraph-tauola/Su 12 V7A=      &    2.206            &        1457.84          \\
\verb=/ZZJetsTo2L2Q_TuneZ2star_8TeV-madgraph-tauola/Su 12 V7A=      &     2.4487          &        790.921          \\
\hline\hline
\end{tabular}
\caption{\label{tab:frEstimatedMCSamples} MC datasets that do not contribute to MC Pred.  The contribution to the background from these processes is covered by data-driven methods, but expected yields based on simulation are nevertheless provided as a reference.
Predicted yields from the SM samples listed here are used directly in the analysis. 
The common part of each dataset name {\tt Summer12\_DR53X-PU\_S10\_START53\_V7A} is replaced with a shorthand {\tt Su12 V7A}. 
All datasets are in the AODSIM data tier.}
\end{center}
\end{sidewaystable}

\clearpage


\begin{sidewaystable}[H]
\begin{center}
\begin{tabular}{l}
\hline\hline
Name                                                                                                                              \\ \hline
\verb=/QCD_Pt_20_MuEnrichedPt_15_TuneZ2star_8TeV_pythia6/Su 12 V7A-v3=                       \\
\verb=/QCD_Pt-15to20_MuEnrichedPt5_TuneZ2star_8TeV_pythia6/Su 12 V7A-v2=                       \\
\verb=/QCD_Pt-20to30_MuEnrichedPt5_TuneZ2star_8TeV_pythia6/Su 12 V7A-v1=                        \\
\verb=/QCD_Pt-30to50_MuEnrichedPt5_TuneZ2star_8TeV_pythia6/Su 12 V7A-v1=                        \\
\verb=/QCD_Pt-50to80_MuEnrichedPt5_TuneZ2star_8TeV_pythia6/Su 12 V7A-v1=                     \\
\verb=/QCD_Pt-80to120_MuEnrichedPt5_TuneZ2star_8TeV_pythia6/Su 12 V7A-v1=                       \\
\verb=/QCD_Pt-120to170_MuEnrichedPt5_TuneZ2star_8TeV_pythia6/Su 12 V7A-v1=                       \\
\verb=/QCD_Pt-5to15_TuneZ2star_8TeV_pythia6/Su 12 V7A-v1=                        \\
\verb=/QCD_Pt-15to30_TuneZ2star_8TeV_pythia6/Su 12 V7A-v2=                      \\
\verb=/QCD_Pt-30to50_TuneZ2star_8TeV_pythia6/Su 12 V7A-v2=                      \\
\verb=/QCD_Pt-50to80_TuneZ2star_8TeV_pythia6/Su 12 V7A-v2=                        \\
\verb=/QCD_Pt-80to120_TuneZ2star_8TeV_pythia6/Su 12 V7A-v3=                     \\
\verb=/QCD_Pt-120to170_TuneZ2star_8TeV_pythia6/Su 12 V7A-v3=                       \\
\verb=/QCD_Pt-170to300_TuneZ2star_8TeV_pythia6/Su 12 V7A-v2=                        \\
\hline\hline
\end{tabular}
\caption{\label{tab:frQCDMCSamples} MC datasets that are used in calculated the Fake Rate used in the closure tests for the purpose of determining the systematic uncertainty on the method as well as correcting the central value of the prediction. 
The common part of each dataset name {\tt Summer12\_DR53X-PU\_S10\_START53\_V7A} is replaced with a shorthand {\tt Su12 V7A}. 
All datasets are in the AODSIM data tier.}
\end{center}
\end{sidewaystable}
\clearpage




Table ~\ref{tab:bEstimatedMCSamples} contains a list of Monte Carlo samples that are used for reference only to help gain insight into  the estimates from the method that predicts the contribution of events that have b-tags from radiation.
\begin{sidewaystable}[H]
\begin{center}
\begin{tabular}{lcc}
\hline\hline
Name                                                                                                                             & Cross section, pb & Luminosity, \fbinv \\ \hline
\verb=/WZJetsTo3LNu_TuneZ2_8TeV-madgraph-tauola/Su 12 V7A=      &       1.0575        &        1908.25        \\
\verb=/ZZJetsTo4L_TuneZ2star_8TeV-madgraph-tauola/Su 12 V7A=      &         0.176908      &         27177.4         \\
\verb=/WWGJets_8TeV-madgraph/Su12 V7A=                                                            & 0.528             &        407.426         \\ 
\verb=/WWWJets_8TeV-madgraph/Su12 V7A=                                                            & 0.08217           &     2737.02           \\ 
\verb=/WWZNoGstarJets_8TeV-madgraph/Su12 V7A=                                              & 0.0633            &        3832.94        \\ 
\hline\hline
\end{tabular}
\caption{\label{tab:bEstimatedMCSamples} MC datasets that do not contribute to MC Pred.  The contribution to the background from these processes is covered by data-driven methods (b-tag estimation from radian jets), but expected yields based on simulation are nevertheless provided as a reference.
Predicted yields from the SM samples listed here are used directly in the analysis. 
The common part of each dataset name {\tt Summer12\_DR53X-PU\_S10\_START53\_V7A} is replaced with a shorthand {\tt Su12 V7A}. 
All datasets are in the AODSIM data tier.}
\end{center}
\end{sidewaystable}

\clearpage


\begin{sidewaystable}[H]
\begin{center}
\begin{tabular}{lcc}
\hline\hline
Name                                                                                                                             & Cross section, pb & Luminosity, \fbinv \\ \hline
\verb=/DYJetsToLL_M-50_TuneZ2Star_8TeV-madgraph-tarball/Su12-v1=    &        3532.8149     &         8.62188         \\ 
\verb=/WZJetsTo3LNu_TuneZ2_8TeV-madgraph-tauola/Su 12 V7A=      &        1.0575       &              1908.25    \\
\hline\hline
\end{tabular}
\caption{\label{tab:bRateComparison} MC datasets used to validate the b-tag content in a di-lepton and tril-lepton sample against each other as a demonstration of the validity of the method used to predict contribution to the background via b-tags that come from radiation. Note that the DY sample differs from that used in the rest of the analysis. This one was specifically chosen because it treats the b-quark's mass the same as in the used WZ sample.
The common part of each dataset name {\tt Summer12\_DR53X-PU\_S10\_START53\_V7A} is replaced with a shorthand {\tt Su12 V7A}. 
All datasets are in the AODSIM data tier.}
\end{center}
\end{sidewaystable}


\begin{sidewaystable}[H]
\begin{center}
\begin{tabular}{lcc}
\hline\hline
Name                                                                                                                             & Cross section, pb & Luminosity, \fbinv \\ \hline
\verb=/TTZJets_8TeV-madgraph_v2/Su 12 V7A=                                         &        0.2057     &        1021.68          \\ 
\verb=/ttbarZ_8TeV-Madspin_aMCatNLO-herwig/Su 12 V19=                    &        0.2057       &                  \\
\hline\hline
\end{tabular}
\caption{\label{tab:ttZGeneratorMCs} MC datasets used to determine the signal systematic uncertainty due to the type of generator used. They are both normalized to the same cross section. Madgraph is an LO generator while aMC@NLO is an NLO generator. The madgraph sample is the same as the one listed in table~\ref{tab:IrreducibleMCSamples}.
The common part of each dataset name {\tt Summer12\_DR53X-PU\_S10\_START53\_V7A} is replaced with a shorthand {\tt Su12 V7A}. 
All datasets are in the AODSIM data tier.}
\end{center}
\end{sidewaystable}

\clearpage

	
\chapter{Event Selections}
\label{ch::EventSelections}
Some intro here...

\section{Triggers for Data Acquisition}
\label{sec:Triggers}
As the desired final state contains 3 isolated leptons (electrons or muons), a set of triggers without pre-scaling may be used that has 1, 2, or 3 isolated leptons as the trigger requirements. The ideal trigger would have moderate \pt thresholds and would not have restrictions on $\eta$ so that leptons from anywhere in the tracker may be counted. Isolation requirements are allowed provided they are not tighter than the offline requirements of XXXXX (see Section~\ref{}). Further a parallel set of single lepton triggers must exist in order to select data measure the ratio used in the background subtraction method which estimates non-prompt lepton contamination (i.e. Fake Rate) in the signal (see Section~\ref{}). \\
Thus a variety of triggers were chosen that require 2 or more leptons (double lepton triggers). Triggers that require 1 or more lepton or that require 3 or more leptons were rejected as they would not match the triggers used to estimate the Fake Rate. The double lepton triggers are assumed to be 100\% efficient. The efficiency to trigger on 1 lepton is very high, and thus with 3 leptons, the odds that at least 2 of them will pass is approximately 100\%. A small systematic uncertainty is added to account for any potential inefficiency (see Section~\ref{}).
	
\begin{table}[h]
\begin{center}
\caption{\small\label{tab:AnalysisTriggers} Triggers used while measuring the yields in data for the main analysis selections. In the trigger name the ``v" at the end of the line stands for a number of versions that changes while data is being collected.}
\begin{tabular}{l|l} \hline \hline
Lepton Type & Trigger  \\ \hline
$\mu\mu$     & \verb=HLT_Mu17_Mu8_v= \\
$\mu \mu$    &\verb=HLT_Mu17_TkMu8_v= \\
ee                   & \verb=HLT_Ele17_CaloIdT_CaloIsoVL_TrkIdVL_TrkIsoVL_v= \\
ee                   & \verb=Ele8_CaloIdT_CaloIsoVL_TrkIdVL_TrkIsoVL_v=\\
$\mu$e          & \verb=HLT_Mu17_Ele8_CaloIdT_CaloIsoVL_TrkIdVL_TrkIsoVL_v= \\
$\mu$e          & \verb=HLT_Mu8_Ele17_CaloIdT_CaloIsoVL_TrkIdVL_TrkIsoVL_v= \\
\hline
\end{tabular}
\end{center}
\end{table}

Events are required to fire a trigger in Data only. Monte Carlo events are not required to fire a trigger. Monte Carlo triggers are not well enough simulated to accurately predict the behavior of Data triggers. This is not a problem as the Data triggers used are un-prescaled and highly efficient.




\section{General Event Cleanup and Vertex Selection}
\label{sec::EventCleanup}

Events in data and simulation are required to pass the following:

\begin{itemize}
\item Scraping cut: if there are $\geq$ 10 tracks, require at least 25\% of them to be high purity. 
\item Require at least one good primary vertex (PV), and use the first such vertex found as a reference point for further selections.  A good vertex is selected by requiring:
	\begin{itemize}
	\item not fake, ?????????
	\item ndof $>$ 4,
	\item $|\rho| < 2$ cm,
	\item $|z| < 24$ cm.  
	\end{itemize}
\end{itemize}
	 
	 
\section{High Level Event Identification}	 
\label{sec:EventSelections}
Good, isolated electrons are selected following the ``medium" identification requirements recommended by the Egamma POG~\cite{eleICHEP2012twiki}.
Electrons are required to have no missing expected inner hits~\cite{conv} and to not be reconstructed as part of a good conversion vertex~\cite{hwwsmurf} so as to suppress background from converted photons.
Additionally, electrons are required to have $\Delta R >0.1$ with respect to any muon passing the selections above.
We reject electrons found in the transition region.
The isolation follows the POG recommendation~\cite{egammapfisotwiki}, using particle-flow based isolation with a cone size of \DR\  $=$ 0.3.  
Subtraction for PU is performed by removing a term defined by the product of the average event energy density and the effective area of the isolation cone from the neutral isolation sum~\cite{egammaisorhoaeff}.
The isolation relative to the electron \pt\ is required to be less than 0.09.\\

Event selections were designed to be middle ground - a compromise between strong background rejection and inclusiveness. It is done in 5 main steps:
\begin{enumerate}
\item Identify a Z candidate consisting of two isolated high \pt \ leptons (20 \GeV ) with opposite electric charge and of the same flavor that are within a $\pm 10$ \GeV window of the Z invariant mass (\zmass);
\item Identify a third lepton which could be the result of a W decay passing the same identification and isolation requirements as the Z leptons;
\item Reject events with a 4th lepton passing a looser set of selections (identification loosened to the standard EGamma medium working point and Muon tight working point and $\pt > 10 \ \GeV$) to be exclusive with the 4 Lepton channel;
\item Identify at least 4 Jets to be consistent with the number from a semi-leptonic $t\bar{t}$ decay;
\item Identify from the Jets at least 2 that pass the CSV b-Tagging algorithm;
\end{enumerate}	 
	 
\section{electron selections}
\label{sec:ElectronSelections}

Electron candidates are RECO GSF electrons with \pt\ \gt\ 20 \GeV\ and \absetaele\ passing the following requirements recommended by the Egamma POG~\cite{eleICHEP2012twiki}:
\begin{itemize}
\item reject electrons in the transition region $1.4442 < \aeta < 1.556$, where $\eta$ is taken from the super-cluster (SC);
\item electrons have to have $\DR >0.1$ with respect to any muon passing the selections in Appendix~\ref{sec:muID};
\item cut-based medium WP as defined by~\cite{eleICHEP2012twiki};
\item transverse impact parameter of the GSF track with respect to the selected PV to be $<200~\micron$;
\item the $z$ coordinate of the GSF track should be within 2~mm with respect to that of the selected PV;
\item Conversion removal by veto of a good reconstructed conversion vertex.  A conversion vertex is considered good if it has no tracker hits towards the beam, has a fit probability above $10^{-6}$, has a displacement of more than 2~cm, and the  CTF track matching to the electron should be a part of the conversion vertex. No requirement is made on the vertex quality flag corresponding to merging and arbitration~\cite{hwwsmurf}.
\end{itemize}

Additionally, the identification contains the following modifications/additions with respect to the above recommendation:

\begin{itemize}
\item the number of missing expected inner hits must be zero~\cite{conv};

\item the $H/E$ is required to be \lt 0.1 (0.075) in the barrel (endcap) to match the requirements in the trigger.
\end{itemize}

The isolation follows the POG recommendation, using particle-flow based isolation with a cone size of \DR  $=$ 0.3~\cite{egammapfisotwiki}.  
In the endcap, an inner veto of \DR $=$ 0.015 (0.08) is imposed for charged hadrons (photons). 
The isolation is corrected for PU by subtracting from the neutral isolation components a term defined by the product of the average event energy density and the effective area of the isolation cone~\cite{egammaisorhoaeff}.  The neutral component after correction is required to be non-negative.  
The isolation relative to the electron \pt\ is required to be less than 0.09.



\section{muon selections}
\label{sec:MuonSelections}
Muons are required to be well identified and isolated, passing the ``tight" identification requirements recommended by the muon POG~\cite{muICHEP2012twiki}. Additionally veto deposits are required to be consistent with a minimum ionizing particle and the d0 requirement is tightened to reduce non-promt lepton contamination.  
Isolation follows the POG recommendation, using  particle-flow based isolation with a $\Delta\beta$ correction for PU and a threshold of 0.1.
A smaller cone size of \DR\ $=$ 0.3 is adopted due to the high hadronic activity expected in signal-like events.\\

Muon candidates are RECO muon objects with \pt\ \gt\ 20 \GeV\ and \absetamu\ and passing the ``tight" identification requirements recommended by the muon POG~\cite{muICHEP2012twiki}:
\begin{itemize}
\item is a global muon;
\item is a particle flow muon;
\item $\chi^2$/ndof of global fit $<$ 10;
\item at least 6 layers with hits in the tracker;
\item the global fit has to include at least one valid hit in the muon subdetectors;
\item there are muon segments in at least two muon stations. Note, this implies that the muon is also an arbitrated tracker muon~\cite{swguidetrackermuonstwiki};
\item at least one pixel hit.
\end{itemize}

Additionally, the identification contains the following modifications/additions with respect to the above recommendation:

\begin{itemize}
\item transverse impact parameter of the silicon (inner) track with respect to the selected PV to be $<200~\micron$;
\item the inner track $z$ should be within 1~mm from the selected PV;
\item ECAL veto deposit \lt\ 4~\GeV\ (veto deposit corresponds to the sum of $E_T$ in the region of the calorimeter
associated with the muon impact);
\item HCAL veto deposit \lt\ 6~\GeV.
\end{itemize}

The isolation follows the POG recommendation, using particle-flow based isolation with a $\Delta\beta$ correction for PU.  
However, a smaller cone size of \DR\ $=$ 0.3 is adopted due to the high hadronic activity expected in signal-like events.  
The isolation calculated using 
$$
\relIso = [ \Sigma_{\rm ch} + {\rm max}(0, \Sigma_{\rm nh} + \Sigma_{\rm ph} - 0.5 \Delta\beta) ]/\pt,
$$
where $\Sigma_{\rm ch, nh, ph}$ are the sums of \pt\ of the charged hadron, neutral hadron, and photon particle flow candidates, respectively.
Here the charged hadrons are matched to the PV and a 0.5~\GeV\ threshold is applied on neutral hadrons and photons.  
The $\Delta\beta$ correction is det	``ermined from the sum \pt\ of charged hadrons not matched to the PV with a threshold of 0.5~\GeV\ in a cone of the same size as the isolation.  
 
The \relIso\ is required to be less than 0.1.


\section{Z-boson lepton pair and 3rd lepton selection}
\label{sec:3LepSelection}
All 3 selected leptons must pass the Identification and Isolation requirements as well as \pt and $\eta$ requirements listed in Sections \ref{} and \ref{}. Furthermore, 2 of the leptons are required to be consistent with a reconstructed Z boson. This includes:
\begin{itemize}
\item opposite charge
\item same flavor
\item invariant mass between 81 and 101 GeV
\item both leptons matched to the same vertex
\end{itemize}
In the case of events where all three leptons are the same flavor and two combinations exist which fit in the invariant mass window, the one which is closest to the measured Z mass of \zmass \ ~\cite{pdg} is chosen.\\

\section{4th lepton veto}
\label{sec:4thLeptonVeto}
To be accepted as an event, three tight leptons as described in Appendix ~\ref{sec:muID} and ~\ref{sec:eleID} must be found. A veto on a fourth lepton passing the unmodified POG recommendations used as the base cuts above is applied. In addition, the 4th lepton has a relaxed \pt \ requirement of 10 \GeV. Finally, the same isolation requirements as above are used.

\section{jet selection}
\label{sec:JetSelection}
There must be at least 3 particle-flow jets with $\pt > 30$ GeV and $|\eta| < 2.5$ and at least a fourth particle-flow jet with $\pt > 15$ GeV and $|\eta| < 2.5$. The threshold on the fourth jet is lowered as this one tends to be produced softer than the other jets.
Jets are reconstructed with the anti-$k_{T}$ algorithm with parameter R = 0.5.  
Jets in simulation have L1FastJetL2L3 (FastJet-based offset correction followed by L2 and L3 corrections) corrections applied.  
Jets in data additionally have L2L3 residual corrections applied~\cite{jetcorrectionstwiki}. 
Selected jets must pass loose {\tt pfJetId} and be separated by $\Delta R >$ 0.5 from any lepton passing the selections above. Finally, an extra pile up rejection is applied where at least 10\% \ of the energy in the jet must be from within a $dZ < 0.05 cm$ of the primary vertex.

\section{b-tag selection}
\label{sec:bTagSelection}

Include details about tagger efficiency and purity as well as some basic info about how it works???????????\\

At least two jets passing the above jet selections in \ref{} and $\pt > 30$ are required to be tagged using the CSV algorithm. Both must be at least passing the loose threshold, and one must be at least passing the medium threshold.
This tagger identifies jets with discriminant larger than  0.244 as Loose b-tagged and 0.679 as Medium b-tagged~\cite{btagICHEP2012twiki}.




\section{optimization}
\label{sec:Optimization}

While defining the event selections, a number of options were considered. All of the selections were chosen to simultaneously reduce the  background compared to signal and the projected error on the cross section using the \ttZ \ signal Monte Carlo and full analysis background estimates. This includes \Ht \ cuts, \met \ cuts, alternative thresholds and multiplicities for b-tagging, alternative  \pt cuts and multiplicities for jets, and varying sized Z mass windows. In the end, too many backgrounds have real \met \ from a $W \rightarrow \ell \nu$ decay for this to be a distinguishing variable. Although \Ht \ can be used to reduce many of the backgrounds, it ended up performing not quite as well as and being redundant with the (related) pure jet cuts used in this note. Additionally, a single tight or medium  b-tag was found to let in too much background, while using 2 mediums or a medium and a tight killed too much signal.  

Optimizing the Z mass window ends up being less straightforward than the other relevant selections. The Z mass window makes little difference in the error on the cross section and was thus chosen by its expected signficance (Table ~\ref{tab:zmass_errorsig}). In Table~\ref{tab:zmass_errorsig}, the significance continues to increase with an even narrower Z mass window than the chosen $\pm 10$ \GeV \ window, but there is a strong reason to stop at $\pm 10$ \GeV.\\

These predictions are made with the assumption that no new systematic uncertanties are introduced. This assumption does not hold up for a narrow enough Z mass window. By comparing the yields of the \ttZ \ signal MC to a 2 lepton Z selection in data, the yields do not scale the same. The 2 lepton Z selection is the same as used to calculate the rate of b-tagged jets in the background estimate of non-top process. Table ~\ref{tab:zmass_massscaling} shows the \ttZ scaling and the 2 lepton data scaling both with a b-veto applied as well as a 2 b-tag selection applied. The reference yields are from the $\pm 15$ \GeV \ Z mass window. The 2 lepton Z data selection is chosen because it is high in statistics and not part of our signal region (yet contains a reconstructed Z boson).

As the window gets narrower beyond $\pm 10$ \GeV, the difference in scaling of the yields in Table ~\ref{tab:zmass_massscaling} becomes significant at approximately a difference of 4-5\%. These narrower mass windows would lead to challenges in estimating the systematic uncertainty introduced by the mass window as well as reduce the significance of that mass window. To avoid this extra difficulty, the $\pm 10$ \GeV \ window is chosen.

\begin{table}[ht!]
\caption{\small \label{tab:zmass_errorsig} Expected signficance and estimated \% error on the cross section by varying the Z mass window. Significance and error estimates are determined with full analysis selections and background estimates in conjunction with the \ttZ \ signal Monte Carlo.}
\begin{center}
\begin{tabular}{c|ccc}\hline
Mass Window (\GeV)   &  Expected Significance  & \multicolumn{2}{c}{Estimated Error on Cross Section}    \\
                     &                         & +         & -                                           \\
\hline \hline
$\pm 15$             &  2.27                   &  +51\%    &  -45\%                                      \\
$\pm 12.5$           &  2.42                   &  +50\%    &  -43\%                                      \\
$\pm 10$             &  2.44                   &  +50\%    &  -43\%                                      \\
$\pm 7.5$            &  2.57                   &  +49\%    &  -42\%                                      \\
$\pm 5$              &  2.51                   &  +51\%    &  -43\%                                      \\
$\pm 2.5$            &  2.21                   &  +60\%    &  -49\%                                      \\
\hline
\end{tabular}
\end{center}
\end{table}


%% \begin{table}[ht!]
%% \caption{\small \label{tab:zmass_expsig} Expected significance by varying the Z mass window. Expected significance is determined with full analysis selections and background estimates in conjunction with the \ttZ \ signal Monte Carlo.}
%% \begin{center}
%% \begin{tabular}{c|c}\hline
%% Mass Window (\GeV)  & Expected Significance            \\
%% \hline \hline
%% $\pm 15$            &  2.27 \\
%% $\pm 12.5$          &  2.42 \\
%% $\pm 10$            &  2.44 \\
%% $\pm 7.5$           &  2.57 \\
%% $\pm 5$             &  2.51 \\
%% $\pm 2.5$           &  2.21 \\

%% \hline
%% \end{tabular}
%% \end{center}
%% \end{table}


\begin{sidewaystable}[ht!]
\caption{\small \label{tab:zmass_massscaling} Scaling of yields in high statistics Z selection with Z mass window. The ``2 Lepton Data'' columns are important numbers in the calculation of the rate of b-tags and are used here because they are a high statistics selection of a Z boson. Statistical errors on the absolute yields in the \ttZ MC \ are between 5 and 6\%. Statistical errors on the DY MC are negligable while statistical errors on the 2 Lepton Data selection are between 0.5\% and 2\% where the DY MC with b-veto is clustered at the low end.}
\begin{center}
\begin{tabular}{c|ccccc}\hline
Mass Window (\GeV)  & \ttZ \ MC   & 2 Lepton DY MC  & 2 Lepton Data  & 2 Lepton DY MC  & 2 Lepton Data  \\
                    &             & with b-Veto     & with b-Veto    & with 2 b-tags   & with 2 b-tags \\
\hline \hline
$\pm 15$            &  100\%      &  100\%          & 100\%          & 100\%           & 100\%  \\
$\pm 12.5$          &  99.5\%     &  98.6\%         & 98.1\%         & 99.0\%          & 98.1\% \\
$\pm 10$            &  97.8\%     &  96.0\%         & 95.6\%         & 97.1\%          & 95.4\% \\
$\pm 7.5$           &  95.2\%     &  91.9\%         & 90.8\%         & 91.9\%          & 91.4\% \\
$\pm 5$             &  87.9\%     &  83.6\%         & 81.4\%         & 85.5\%          & 82.0\% \\
$\pm 2.5$           &  65.3\%     &  61.5\%         & 56.8\%         & 60.9\%          & 57.5\% \\

\hline
\end{tabular}
\end{center}
\end{sidewaystable}


\chapter{Background estimation methods}
	\section{irreducible}
        		\subsection{re-describe why they are irreducible}
        		\subsection{chosen cross sections and errors on cross sections}
	\section{fake rate}
        		\subsection{re-describe fakes, discus sources}
       		\subsection{overview of method}
        		\subsection{Fake rate data sets and event selection}
        		\subsection{contamination from electro-weak processes correction}
        		\subsection{fake rates for electrons and muons}
	\section{b rate}
        		\subsection{re-describe source of b-tags not from a top}
        		\subsection{overview of method}
        		\subsection{sources of uncertainty on method}
        		\subsection{measured b rate}

\chapter{Efficiencies and Uncertainty}
\label{ch:eff_and_unc}

Systematic uncertainties on signal event selections arise from differences between simulated events and the actual performance of  the detector or slight differences in physical processes from simulated processes.

A summary of systematic uncertainties is given in Table~\ref{tab:systSumm}. Full descriptions of each systematic are presented in the following sub-sections.

\begin{table}[h]
\begin{center}
\caption{\small\label{tab:systSumm}Summary of systematic uncertainties on the signal selection and
expectation. 
Reported values are fractional, relative to the total cross section.}
\begin{tabular}{lcccc}\hline
Source 					& Method & Total Systematic 	\\ \hline
Jet Energy Scale			& Momentum Scale Up/Down                        & 4.8\%	\\
Jet Energy Resolution	                   	& Momentum Smearing                                         & 0.4\%	\\

b-tag (light flavor)                          & Discriminant Re-weight                              & 1.0\%       	\\
b-tag (b flavor)		                      & Discriminant Re-weight                              & 2.9\%	\\	
Q$^2$                                         & Q$^2$ Scale Up/Down                                 & 1.7\% \\
Matching                                      & Matching Scale Up/Down                              & 1.2\% \\
Top Mass                                      & Mass Scale Up/Down                                  & 2.5\% \\
PDF				                              & PDF Re-weight                                       & 1.5\%	\\
Generator                                     & Compare 2 ttZ Samples                               & 5.0\% \\
Pile Up                                       &                                                     & 1.0\% \\
Trigger                                       &                                                     & \lt 1\% \\
Lepton Identification, Isolation,             & Tag \& Probe                                        & 6.2\% \\
and Event Composition  & & \\
\hline
Total 					                                                                             & & 10.5\% 	\\
\hline
\end{tabular}
\end{center}
\end{table}

\section{Lepton MC to Data Scale Factors and Uncertainties}
\label{sec:tag_and_probe}

Data-Monte Carlo scale factors are derived using the leptonic identification and isolation requirement efficiencies measured with the ``tag and probe" method. The method identifies dilpeton Z events from the full 2012 dataset. One lepton, known as the ``tag," passes the complete set of lepton selections. The other lepton, known as the ``probe," is allowed to pass a relaxed set of requirements. In the case of measuring the efficiency of the isolation requirement, the probe is required to pass the full identification and quality cuts but not the isolation. The efficiency is the ratio of probes passing the isolation requirement to those not passing the requirement. In the case of measuring the efficiency of the identification requirement, the probe is required to pass the full isolation cut but not the identification cuts. \\

For electrons, the tag is required to be matched to the \verb=HLT_Ele27_WP80= trigger, which requires one well-identified electron passing the WP80 electron ID~\cite{egammaidtwiki} with \pt \gt 27 \GeV. The tag electron must also pass the electron identification requirements and isolation cut in Sec~\ref{sec:ElectronSelections}. The \pt \ threshold of the tag electron is additionally raised to 32 \GeV \ to avoid trigger turn-on effects. The probe electron has a base requirement of
\begin{itemize}
\item \pt \gt 10 \GeV 
\item $|\eta| \lt 2.4$, and excluding electrons with a supercluster between $1.4442 \lt |\eta| \lt 1.566$
\end{itemize}
The identification efficiency is measured with the probe electron additionally passing the isolation requirements described in Sec~\ref{sec:ElectronSelections} but not the identification requirements. The derived efficiency is directly applicable to the leptons in the full tri-lepton analysis as identification requirements are a property of the lepton alone. The isolation efficiency, however, is dependent on the energy (mainly energy from hadronic activity) in the event. As such, both the electron identification and an additional jet selection is required matching those described in Sec~\ref{sec:JetSelection}. The overall electron efficiency may be measured by relaxing the probe electron to the nominal value. The electron efficiencies are summarized in Table~\ref{tab:eleffiency}. A simultaneous fit is performed on the di-lepton data in the invariant mass range of $60-120 \ \GeV$ using models from Table~\ref{tab:tnpmodels} selected for performance on a bin-by-bin basis. This becomes necessary as the kinematics varies with \pt \ and \aeta \ bins and fitting models must be chosen for best results.\\

\begin{table}[h]
\begin{center}
\caption{\small\label{tab:tnpmodels} Models used for fitting the signal or the background contribution in the tag and probe method.}
\begin{tabular}{l|c}\hline
 Model                                                                               & Usage \\ \hline \hline
 Breit-Wigner function $\star$ Crystal-Ball function & Signal \\
 MC-based template function                                       & Signal \\
 Exponential                                                                    & Background \\
 Exponential $\star$ Error function                             & Background \\
 Polynomial                                                                     & Background \\
 Polynomial $\times$ Exponential function               & Background \\
 Chebyshev Polynomial                                               & Background \\
\end{tabular}
\end{center}
\end{table}

\begin{sidewaystable}[h]
\begin{center}
\caption{\small \label{tab:eleffiency} Measured electron efficiency ratios using the tag and probe method. Errors are statistical only.}
\begin{tabular}{c|c|c|c|c|c} \hline \hline
\pt - $|\eta|$   &                 &20 - 30 \GeV & 30 - 40 \GeV & 40 - 50 \GeV & 50 - 200 \GeV \\ \hline
%                        & MC          &  &  & \\
%                        & Data       & & & \\
0.0 - 0.8         & Data/MC & 0.947 +/- 0.003 & 0.963 +/- 0.001 & 0.975 +/- 0.001 & 0.973 +/- 0.001 \\ \hline
%                        & MC          & & & \\
%                        & Data       & & & \\
0.8 - 1.4442   & Data/MC & 0.885 +/- 0.004 & 0.942 +/- 0.001 & 0.960 +/- 0.001 & 0.962 +/- 0.001 \\ \hline
%                         & MC          & & & \\
%                         & Data       & & & \\
1.566 - 2.0      & Data/MC & 0.928 +/- 0.016 & 0.936 +/- 0.002 & 0.959 +/- 0.001 & 0.969 +/- 0.002\\ \hline
%                         & MC          & & & \\
%                         & Data       & & & \\
2.0 - 2.4          & Data/MC & 0.994 +/- 0.006 & 0.980 +/- 0.003 & 0.982 +/- 0.002 & 0.978 +/- 0.004 \\ \hline \hline
\end{tabular}
\end{center}
\end{sidewaystable}

Muon identification and isolation requirement efficiencies are measured using the same methods as with the electrons. Appropriate muon specific identification and isolation requirements as described in Sec~\ref{sec:MuonSelections} are used instead of the electron specific ones above.\\

\begin{sidewaystable}[h]
\begin{center}
\caption{\small \label{tab:mueffiency} Measured muon efficiency ratios using the tag and probe method. Errors are statistical only.}
\begin{tabular}{c|c|c|c|c|c} \hline \hline
\pt - $|\eta|$ &                 &20 - 30 \GeV & 30 - 40 \GeV & 40 - 50 \GeV & 50 - 200 \GeV \\ \hline
%                      & MC          &  &  & \\
%                      & Data       & & & \\
0.0 - 1.20     & Data/MC & 0.962 +/- 0.001 & 0.972 +/- 0.001 & 0.978 +/- 0.000 & 0.974 +/- 0.001 \\ \hline
%                      & MC          & & & \\
%                      & Data       & & & \\
1.20 - 2.50   & Data/MC & 0.974 +/- 0.002 & 0.978 +/- 0.001 & 0.984 +/- 0.000 & 0.978 +/- 0.001 \\ \hline \hline
\end{tabular}
\end{center}
\end{sidewaystable}

\clearpage

For both the electrons and muons, the Tag and Probe measurements were repeated in two ways. The first measured the scale factor by allowing the probe to fail both ID and ISO requirements. The second allowed the probe to fail ID or ISO requirements separately, and the two scale factors were multiplied together. The difference in scale factor between those derived by allowing both ID and ISO to fail together and those derived by combining two measurements that allowed ID and ISO to fail individually is used to assess a systematic uncertainty in this procedure. An average value of deviation across the bins is chosen. In this case, we use 1.5\% on electrons and 0.3\% on muons in addition to the statistical errors quoted in the tables.

Additionally, the  measurements are performed in a pure Drell-Yan sample and may differ slightly for the isolation requirements from those in the analysis selections due to hadronic activity. A study in pure MC was performed to compare the isolation values of electrons or muons generator matched to a status three electron or muon. A Drell-Yan sample and a \ttZ \ sample were chosen for this comparison and listed in Appendix~\ref{sec:mc_details}. The isolation curves are compared for leptons from both samples by looking at the fraction of events passing the isolation cut. This is done by comparing DY sample with no Jet multiplicity requirement to a \ttZ \ sample requiring four Jets.  The difference between the fractions  additional systematic is assessed due to the difference in efficiency for the respective analysis isolation cuts between the two samples. Leptons are required to pass the acceptance selections, $\pt > 20 \GeV$, and full identification selections as defined in Sec~\ref{sec:EventSelections}. For muons and electrons the difference is 2\%.

Finally, the scale factors in Tables~\ref{tab:eleffiency} and~\ref{tab:mueffiency} are used to reweight MC events based on the identified leptons. To determine the total amount of uncertainty due to the lepton efficiencies, the scale factors are varied up and down by their total errors (stat and systematic). The electron errors and muon errors are assumed to be uncorrellated and thus varied independently between the two. The hadronic activity error is considered correlated for both flavors and varried at the same time for both flavors. The difference between the yields of the up and down variations is used to determine a systematic uncertainty on lepton efficiency, and the three variations (e, $\mu$, hadronic) are added in quadrature. This uncertainty is 6.2\% and is summarized in Table~\ref{tab:systSumm}.\\\\





\section{Systematic Uncertainty Due to Triggers}   
\label{sec:trigger_syst}
Dilepton triggers are used with a selection that ultimately chooses three leptons. This means that the triggers are ultra-efficient as only 2/3 of the leptons need to be identified by the triggers. We assign a 1\% uncertainty to cover the nearly negligible chance of a three lepton event failing a dilepton trigger.

\section{b-tagging Efficiency and Associate Errors}
\label{sec:btag_syst}
b-tagged jets are chosen based on CSV discriminant thresholds. Differences arise in the shape of the discriminant distributions in Data and MC. In the past event weights were scaled to account for this and match up the number of b-tags in Data and MC. Another current method involves promoting or demoting b-tagged jets (e.g. CSVL to None or CSVM to CSVT) using random numbers. This method has the drawback that it becomes fairly complicated to perform when using more than one threshold (e.g. requiring one CSVL and one CSVM b-tagged jets). A more natural and fitting method reshapes the MC discriminant distribution. This has the advantage of preferentially promoting or demoting b-jets on the border of a threshold and removing any complexities arising from multiple levels of tightness. The BTV POG recommended methods are summarized for use in reference~\cite{bTagSF}.\\

The discriminant is reshaped based on Data/MC scale factors measured in~\cite{BTV11003} by the BTV POG for both light flavor jets mis-tagging and b-jets tagging. Signal systematics are then determined by reshaping the MC discriminant with the scale factors varied up and down by the error on the scale factor measurement. The MC FlavorAlgo values are used for the jet to quark truth matching. This variation is performed in the \ttZ \ signal sample listed in Sec~\ref{sec:mc_details}. The percentage difference up and down from the central value is then chosen as the error. This is done separately and independently for both the light flavor scale factors and the b scale factors. The two errors are added in quadrature to get a total (see Table~\ref{tab:systbTag}).


\begin{table}[h]
\begin{center}
\caption{\small\label{tab:systbTag} Summary of systematic b-tag uncertainties split by light flavor and b contributions.}
\begin{tabular}{lc}\hline
Source & Total Systematic \\ \hline
b Quarks & 2.9\% \\
Light Flavor Quarks & 1.0\% \\ \hline
Total & 3.1\% \\
\hline
\end{tabular}
\end{center}
\end{table}



%\chapter{Other Uncertainties Due to Use of Simulations}
\section{Uncertainties  Involved in Event Generation}
The previous chapter was concerned with uncertainties due to reconstructed and identified quantities in the MC simulations. The following chapter will discuss uncertainties in the simulation at a lower level, at the level of event generation. These can come from the parton distribution functions used, masses used for particles, or the order to which calculations are performed.\\ 

\subsection{Uncertainties Due to Parton Distribution Functions}	
Proton-proton interactions are not as straightforward as single particle interactions (e.g. electron-electron). Although the theory for quarks interacting or gluons interacting is well understood, interaction between composite objects are much more complicated. Parton Distribution Functions enter to help describe the probability density of finding a particular particle inside the proton with a certain longitudinal momentum fraction at a specific resolution (Q$^2$). Due to the non-perturbative nature of the partons which are not seen as free particles, this cannot be calculated via perturbative QCD. External methods must be used to probe the structure and distribution of the proton. This measured structure can then be used within a theoretical framework. Experimental error must be assessed to determine the impact of choosing different PDFs for the MC simulations.\\

The recommendations of the PDF4LHC~\cite{PDF4LHC} group are followed for estimating the acceptance uncertainty from the PDF. The predictions from CT10, MSTW, and NNPDF sets are used. The LHAPDF package is used for PDFs and the re-weight function. Each set and all of its error subsets are run over and used to re-weight the \ttZ \ MC to determine an acceptance. MC events are re-weighted from the generator PDF (CTEQ6) to the variation based on the parton type and momentum.\\

\subsubsection{CT10}
The PDF uncertainty for CT10 is calculated by

\begin{equation}
\Delta A^{+} = \sqrt{ \displaystyle \sum \limits_{i=1}^N [	\max(A_i^{+} - A_0, A_i^{-} - A_0, 0) ] ^ 2},
\end{equation}

\begin{equation}
\Delta A^{-} = \sqrt{ \displaystyle \sum \limits_{i=1}^N [	\max(A_0 - A_i^{+}, A_0 - A_i^{-} , 0) ] ^ 2},
\end{equation}
where $A_0$ is computed with the central PDF, i runs over the error sets, $A^{+}$ and $A^{-}$ denote the up and down errors respectively on the acceptances of the ith eigenvector subset. The $\alpha _s$ uncertainty is obtained by considering PDFs for $\alpha _s$ = 0.116, 0.117, 0.118, 0.119, 0.120. The same equations above are applied to these subsets. Finally, it is noted that the PDFs for CT10 are at 90\% CL variations and thus the results are scaled down by 1.645 to obtain the 68\% CL uncertainty.
	
\subsubsection{MSTW}
Five different sets are used for MSTW with $\alpha _s = \alpha _s ^ 0$, $\alpha _s ^ 0 \pm 0.5 \sigma$, and $\alpha _s ^ 0 \pm \sigma$. The uncertainty is given with the following formulae:
\begin{equation}
\Delta A^{+} = \underset{\alpha _s}{\max} [ A_{\alpha _ s} + \Delta A_{\alpha _s} ]  - A_0,  
\end{equation}

\begin{equation}
\Delta A^{-} = A_0 -  \underset{\alpha _s}{\max} [ A_{\alpha _ s} - \Delta A_{\alpha _s} ], 
\end{equation}
where $A_0$ is from the central value PDF.\\

\subsubsection{NNPDF2.0}
The NNPDF samples of PDF+$\alpha _s$ is obtained by using NNPDF sets with different fixed $\alpha _s$ corresponding to a Gaussian sample around the nominal $\alpha _s = 0.119$ as shown in Table~\ref{tab:nnPDFsets}.
 
\begin{table}[h]
\caption{\label{tab:nnPDFsets} NNPDF $\alpha _s$ ranges for determining the impact of $\alpha _s$ ranges. Number of replicas for each $\alpha _s$ are listed.}
\begin{center}
\begin{tabular}{c|ccccccc}\hline
$\alpha _s$         &  0.116 & 0.117 & 0.118 & 0.119 & 0.120 & 0.121 & 0.122 \\ \hline
$N_{rep}$                &  5         &  27      &  72      &   100   &  72     &   27    &    5       \\
\hline
\end{tabular}
\end{center}
\end{table}

The uncertainty is computed using the following equations:

\begin{equation}
\Delta A ^{+} = \sqrt{  \frac{1}{N^{+} - 1} \sum \limits_{i=1}^{N^{+}} (A_i - A_0)^2},
\end{equation}
\begin{equation}
\Delta A ^{-} = \sqrt{  \frac{1}{N^{-} - 1} \sum \limits_{j=1}^{N^{-}} (A_0 - A_j)^2},
\end{equation}
with $A_0$ computed front he central value PDF at $\alpha = 0.119$, i running over the $N^{+}$ replicas where $A_i$ \gt $A_0$, and j running over the $N^{-}$ replicas with $A_j$ \lt $A_0$.\\

\subsubsection{Combined Results}
To combine the measurements into one error, the uncertainty is computed with the following equations:

\begin{equation}
\Delta _{\max} = \underset{i}{\max}  [A_i + \Delta A^{+} _i],
\end{equation}
\begin{equation}
\Delta _{\min} = \underset{i}{\min}  [A_i - \Delta A^{-} _i],
\end{equation}
and the PDF uncertainty is
\begin{equation}
\mathrm{Unc}_{PDF} = \frac{1}{2}(\Delta _{max} - \Delta _{min} ),
\end{equation}
where i runs over all of the PDF sets listed above.\\

Using this procedure, yields an uncertainty of 1.5\%.



\subsection{Uncertainties Measured in Top Samples}
A number of uncertainties are measured by varying parameters in top samples. Ideally, these parameters should be varied in a \ttZ \ sample, but since none were produced centrally by CMS, already existing \ttbar samples were used with analogous cuts to the \ttZ \ analysis. that is one lepton was selected, two b-tagged jets, and an additional two jets. These samples are listed in Table~\ref{tab:sampleupdown}.

\begin{table}[h]
\caption{\label{tab:sampleupdown} List of alternate \ttbar \ samples scaling up or down relative parameters to the systematics. Summer12\textunderscore DR53X-PU\textunderscore S10\textunderscore START53\textunderscore V7A-v1 is replaced by SU12 for brevity.}
\begin{center}
\begin{tabular}{l}\hline
Sample   \\ \hline
 \verb=/TTJets_mass178_5_TuneZ2star_8TeV-madgraph-tauola/SU12/AODSIM=   \\
 \verb=/TTJets_mass166_5_TuneZ2star_8TeV-madgraph-tauola/SU12/AODSIM=   \\  %\hdashline
 \verb=/TTJets_scaleup_TuneZ2star_8TeV-madgraph-tauola/SU12/AODSIM=  \\
 \verb=/TTJets_scaledown_TuneZ2star_8TeV-madgraph-tauola/SU12/AODSIM=  \\ %\hdashline
 \verb=/TTJets_matchingup_TuneZ2star_8TeV-madgraph-tauola/SU12/AODSIM=  \\
 \verb=/TTJets_matchingdown_TuneZ2star_8TeV-madgraph-tauola/SU12/AODSIM=  \\
\hline
\end{tabular}
\end{center}
\end{table}

In order to use the \ttbar \ samples, the full analysis selections must be modified to become a mono-lepton selection instead of a tri-lepton selection, so that only prompt leptons are considered and the jet production methods remain the same. In essence this just removes the Z selection. There is an additional benefit from this modified selection in that it creates a high statistics region to derive the systematic errors. The number of events passing the modified selections in each sample are compared to the number of generator level events with 1 lepton. The /TTJets\_SemiLeptMGDecays\_8TeV-madgraph/Su12\_V7A\_ext-v1/AODSIM  (where Su12 stands for Summer12\_DR53X-PU\_S10\_START53) sample is used to produce a central value, and the fraction passing from the varied samples is compared to the fraction passing from the central value to determine the systematic error on each variation. The results are summarized in Table ~\ref{tab:systupdown}.

\subsubsection{Uncertainties Due to Parton Momentum Transfer}
Proton structures can be probed with electrons and the inelastic scattering follows a scaling known as Bjorken scaling. This scaling depends on the momentum transfer produced by the photon force carrier in the interaction. This momentum is in the form of a 4-vector and included in the scaling as a quadratic (written as Q$^2$).\\

\subsubsection{Uncertainties Due to Top Mass}
The top mass is measured as $173.21 \pm 0.51 \pm 0.71$ \GeV~\cite{pdg}. However, due to the size of the error on the mass, it's impact on simulated samples must be evaluated. This mass is varied up and down in simulation production by a range wider than that of the error in order to gauge its impact.\\


\subsubsection{Uncertainties Due to Matching}
Matching is a process that occurs between the Matrix Element simulation of the underlying event and the parton showering afterwards. Generally these two steps are handled by different pieces of software. At NLO level this is difficult because a scheme must be used to prevent double counting. At LO level this process can be highly dependent on jet resolution. Again, the best way to identify the impact on the uncertainty due to matching is by varying the matching scheme in simulated samples.\\


\begin{table}[h]
\caption{\label{tab:systupdown} Summary of systematic b-Tag uncertainties split by light flavor and b contributions. Note: Half the error established by varying the top mass is used because the mass range is much wider than the current considered error on the top mass measurement.}
\begin{center}
\begin{tabular}{lccc}\hline
Source                  &  N Pass / N Gen & \% Deviation from Central & Notes\\ \hline
Central                 & 0.265 & & \\
Q$^2$ Up                 & 0.261 & -1.5\% & \\
Q$^2$ Down           & 0.270 & 1.8\% & \\
Matching Up       & 0.261 & -1.6\% & \\
Matching Down  & 0.267 & 0.8\% & \\
Mass 166.5         & 0.251 & -5.2\% & Use half \\
Mass 178.5         & 0.277 & 4.5\% & Use half \\
\hline
Total                     &             & 3.3\% & Used half of\\
                              &             &             & Mass variation \\
\hline
\end{tabular}
\end{center}
\end{table}



\subsection{Uncertainties Due to Generators}	

The MC samples used in this analysis are primarily generated with the Madgraph event generator. Madgraph produces events at Leading Order. This may lead to a change in acceptance of the events compared to an NLO or NNLO event generator which comes into play with the signal acceptance. Despite the general belief that NLO samples are more accurate, Madgraph samples were still chosen here due to the inconsistent widespread availability of aMC@NLO samples as well as the fact that the available \ttZ \ and \ttW \ aMC@NLO were produced without final state photon radiation. To study this potential uncertainty, two \ttZ samples are used and are summarized in Table~\ref{tab:App:ttZGeneratorMCs}. The first is a Madgraph~\cite{Alwall:2011uj} sample which is used elsewhere in the analysis. The second is an aMC@NLO (~\cite{Frederix:2011zi, Frederix:2011ss} based on the MC@NLO formalism~\cite{Frixione:2002ik} and the MadGraph5 framework~\cite{Alwall:2011uj}) sample. \\




For each sample an efficiency ($\epsilon$) is prepared. The efficiency is defined as
\begin{equation}
\epsilon = \frac{n\ with\ 3\ leptons\ and\ 4\ jets}{n\ with\ 3\ leptons}.
\end{equation}

The exact selections for the denominator are:
\begin{itemize}
\item Exactly three status 3 generator leptons (electrons or muons)
\item All of the above generator leptons with \pt \gt 20 \GeV and \aeta \lt 2.4.
\item All of the above generator leptons came from a W or a Z.
\item Exactly three reconstructed leptons (electrons or muons) that pass the identification and isolation selections in Sec~\ref{sec:EventSelections}
\item Reconstructed leptons that do not match the generator leptons above are rejected.
\end{itemize}

The exact selections for the numerator are:
\begin{itemize}
\item Denominator as above.
\item At least four pfJets with applicable energy corrections with \pt \gt 20 \GeV and \aeta \lt 2.4. 
\item Jets within a cone 0.5 of a lepton identified in the numerator are rejected.
\item Jets with \lt 10\% of their energy coming from the primary vertex are rejected.
\end{itemize}

Then the difference is defined as
\begin{equation}
\Delta = \left| 1 - \frac{\epsilon _{aMC@NLO}}{\epsilon _{Madgraph}} \right|
\end{equation}
and the results are summarized in Table~\ref{tab:systgeneratorsum}

\begin{table}[h]
\caption{\label{tab:systgeneratorsum} Summary of the efficiencies using the Madgraph and aMC@NLO event generators for the jet selections.}
\begin{center}
\begin{tabular}{ccc}\hline
Madgraph           &  aMC@NLO & Difference \\ \hline
0.38                      & 0.36              & 5\%\\
\hline
\end{tabular}
\end{center}
\end{table}

After evaluating and comparing the efficiencies, The contribution to the signal uncertainty from the generator calculation order is set at 5\%.

\section{Jet Energy Scale and Resolution}
Jets are experimental signatures at particle detectors that seek to cluster together showering particles that came from a single source and treats that cluster as a composite object for the sake of the measurements. The accuracy of the energy in this jet are determined by both resolution and scale. The jet energy resolution is a determination of the spread of the distribution of the pull of the energy (pull = true - measured). This resolution is different between data and simulation and simulation must be corrected to be more accurate. This also introduces an uncertainty which must be assessed. Jet energy scaling seeks to address any systematic biases in the jet energy measurement and scales the measured energy by factors to correct it. There is still an associated uncertainty on the jet energy and the contribution of this due to the scale must also be measured.\\

The jet energies have been corrected in data with a residual correction factor to account for differences between measured jet energies in data and in Monte Carlo~\cite{jes_ref}. These corrections come with uncertainties and thus matter for the signal acceptance. To account for this, we vary the jets' transverse momenta up and down by the one standard deviation uncertainties and compare the change in predicted yields in MC. The resultant uncertainty is 4.8\% and is listed in Table~\ref{tab:systSumm}.\\

The recommended prescription for determining uncertainty on the energy resolution of particle flow jets is outlined in~\cite{jer_ref}. Using signal Monte Carlo, the prescription was applied as follows:
\begin{itemize}
\item Each reconstructed jet is matched to the closest generator jet within a cone of $\Delta R < 0.5$.
\item If a match is found the transverse momentum of the reconstructed jet is scaled by.
\begin{equation}
\pt \rightarrow max \left[0, \pt ^{gen} + c \times \left(\pt - \pt ^{gen} \right) \right]
\end{equation}
where $\pt ^{gen}$ is the transverse momentum of the matched generator jet, and c is the data/MC scale factor between the measured and expected particle flow jet resolution in Table~\ref{tab:jer_scalefactor}.
\item If no match is found a gaussian centered at unit and with a width of $c$ is used to smear the momentum instead.
\end{itemize}

\begin{table}[h]
\caption{ \label{tab:jer_scalefactor} Data/MC scale factors used in determining the Jet Energy Resolution. Scale factors are binned in $\eta$ as the detector response varies with $\eta$.}
\begin{center}
\begin{tabular}{c|c}\hline
Jet Pseudorapidity & Scale Factor \\ \hline \hline
0.0 - 0.5 & 1.052 \\
0.5 - 1.1 & 1.057 \\
1.1 - 1.7 & 1.096 \\
1.7 - 2.3 & 1.134 \\
2.3 - 5.0 & 1.288 \\
\hline
\hline
\end{tabular}
\end{center}
\end{table}

To determine the systematic uncertainty on this procedure, up and down variations (Table~\ref{tab:jer_scalefactor_updown}) on the scale factor c are used. The yields in \ttZ \ signal MC are determined with the central value of c as well as the up and down values. The uncertainty is given by
\begin{equation}
\mathrm{uncertainty} = \left| \frac{Yield _{up} - Yield _{down}}{Yield _{central}} \right|.
\end{equation}

\begin{table}[h]
\caption{ \label{tab:jer_scalefactor_updown} data/MC up/down scale factors used in determining the systematic uncertainty on Jet Energy Resolution.}
\begin{center}
\begin{tabular}{c|c|c}\hline
Jet Pseudorapidity & Scale Factor (UP) & Scale Factor (DOWN)\\ \hline \hline
0.0 - 0.5 & 1.115 & 0.990 \\
0.5 - 1.1 & 1.114 & 1.001 \\
1.1 - 1.7 & 1.161 & 1.032 \\
1.7 - 2.3 & 1.228 & 1.042 \\
2.3 - 5.0 & 1.488 & 1.089 \\
\hline
\hline
\end{tabular}
\end{center}
\end{table}

After following this procedure, the acceptance systematic due to Jet Energy Resolution is found to be 0.4\%. This result is summarized in Table ~\ref{tab:systSumm}.\\

\section{Pile Up}
In the LHC, large bunches of protons collide at a rapid rate. The rate of bunch crossings can theoretically be pushed to near the threshold of timing for the machine's electronics which can lead to residual energy still being measured in the detector from a previous collision during the current collision. This is called ``out of time pile up'' and is not an issue at the current rate of collisions for the LHC. Another form of pile up, known as ``in time pile up'' is an issue and is caused by multiple protons colliding within the crossing bunches.\\ 

Proton collisions are frequent, but collisions that produce interesting outcomes are not. So the LHC allows many collisions to happen at once in hopes that one will be interesting. In time pile up occurs when one proton collision produces an outcome that physicists would like to study, but at the same time other collisions produce uninteresting outcomes that nevertheless add energy into the detector which may alter the appearance of the decay products of the collision that physicists want to investigate.\\

Several techniques are employed to mitigate this affect in the jet clustering algorithms, jet energy measurements, and the lepton energy measurements with most of the techniques reducing to some level of subtracting an average ambient energy which has been measured in minimum bias events. However, this still produces some level of uncertainty in the current measurements, and can matter for event acceptance (for example when determining the amount of energy in a cone around a lepton for an isolation measurement).\\

To evaluate this level of uncertainty, Monte Carlo samples are compared to data in the selection region. The number of vertices in the events are measured, and the Monte Carlo events are reweighed to have the same distribution of vertices as the data events. From here total cross section of minimum bias events is varied up and down by 5\% when reweighing the Monte Carlo samples. The effect on the signal yield is found to be $\pm 5\%$ after full selections.



\chapter{Results}
	\section{yields, measured signal}
	\section{top mass reconstruction}
	\section{cross section calculation and significance}
   

% --------------------------------------------------------------------------- %
% --------------------------------------------------------------------------- %
\chapter{Summary and Conclusions}
\label {ch:conclusion}
% --------------------------------------------------------------------------- %
% --------------------------------------------------------------------------- %
This thesis reports on a search for... \\

The future of this analysis looks towards the next LHC run that will increase
the center-of-mass energy to 13 or 14 \TeV...




This dissertation presents the measurement of the cross section of associated production of top anti-top pairs with Z bosons at $\sqrt{s} = 8 \TeV$. The measurement is an inclusive search performed in a tri-lepton final state. In \intLumi of data, the cross section for \ttZ \ production is measured as $\sigma=194 _{-89} ^{+105}$ \ fb with a significance of 2.33. Finally, the ratio of measured to theoretical cross section is $0.94_{-0.43} ^{+0.51}$.

\bibliographystyle{lucas.bst}
\bibliography{include/refs}

\appendix


\end{document}
